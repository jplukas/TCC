\documentclass[11pt]{article}
\usepackage[table]{xcolor}
\usepackage[a4paper, portrait, margin=1.1811in]{geometry}
\usepackage[portuguese]{babel}
\usepackage[utf8]{inputenc}
\usepackage[T1]{fontenc}
\usepackage{helvet}
\usepackage{lipsum}  
\usepackage{etoolbox}
\usepackage{graphicx}
\usepackage{titlesec}
\usepackage{caption}
\usepackage{booktabs}
\usepackage{xcolor} 
\usepackage{url}
\def\UrlBreaks{\do\/\do-}
\usepackage{hyperref}
\hypersetup{
    colorlinks=true,
    linkcolor=blue,
    filecolor=magenta,      
    urlcolor=blue,
    citecolor=blue,
    pdftitle={Proposta TCC João Pedro},
    pdfpagemode=FullScreen,
    }

\urlstyle{same}
\usepackage{caption}
\captionsetup[figure]{name=Figure}
\graphicspath{ {./images/} }
\usepackage{scrextend}
\usepackage{fancyhdr}
\usepackage{graphicx}
\usepackage[T1]{fontenc}
%\pagestyle{plain}
\makeatletter
\patchcmd{\@maketitle}{\LARGE \@title}{\fontsize{16}{19.2}\selectfont\@title}{}{}
\makeatother

\usepackage{authblk}
\renewcommand\Authfont{\fontsize{10}{10.8}\selectfont}
\renewcommand\Affilfont{\fontsize{10}{10.8}\selectfont}
\renewcommand*{\Authsep}{\\}
\renewcommand*{\Authand}{\\}
\renewcommand*{\Authands}{\\}
\setlength{\affilsep}{2em}  
\newsavebox\affbox
\author{\textbf{João Pedro Lukasavicus Silva (IME - USP)}\\
        \textbf{Orientador: Mateus Espadoto (IME - USP)}\\
}
\titlespacing\section{0pt}{12pt plus 4pt minus 2pt}{0pt plus 2pt minus 2pt}
\titlespacing\subsection{12pt}{12pt plus 4pt minus 2pt}{0pt plus 2pt minus 2pt}
\titlespacing\subsubsection{12pt}{12pt plus 4pt minus 2pt}{0pt plus 2pt minus 2pt}


\titleformat{\section}{\normalfont\fontsize{11}{15}\bfseries}{\thesection.}{1em}{}
\titleformat{\subsection}{\normalfont\fontsize{11}{15}\bfseries}{\thesubsection.}{1em}{}
\titleformat{\subsubsection}{\normalfont\fontsize{11}{15}\bfseries}{\thesubsubsection.}{1em}{}

\titleformat{\author}{\normalfont\fontsize{11}{15}\bfseries}{\thesection}{1em}{}

\title{\vspace{-1.0cm} \textbf{\Large Proposta - Trabalho de Formatura Supervisionado}\\
	Modelagem de Tópicos em Textos Históricos utilizando LLMs
 \vspace{0.2cm}
 
\large 2025}
\date{}

\begin{document}

\pagestyle{headings}	
\newpage
\setcounter{page}{1}
\renewcommand{\thepage}{\arabic{page}}

\captionsetup[figure]{labelfont={bf},labelformat={default},labelsep=period,name={Figura }}
\captionsetup[table]{labelfont={bf},labelformat={default},labelsep=period,name={Tabela }}
\setlength{\parskip}{0.5em}
	
\maketitle

\section{Introdução}
Tradicionalmente, pesquisadores das áreas de humanidades como história, filosofia, entre outras, dependem da análise de grandes quantidades de fontes para a realização do seu trabalho. Esta análise tipicamente é realizada de forma manual, e com grande custo, tanto em termos de tempo quanto de recursos humanos, custo este que pode se tornar um fator limitador da produção científica desses pesquisadores.

A área de processamento de linguagem natural (NLP) possui métodos bem estabelecidos para análise semântica de textos, como por exemplo, modelagem de tópicos~\cite{deerwester1990indexing,blei2003latent,jelodar2019latent}. No entanto, o uso destas técnicas requer certa familiaridade com o seu funcionamento, para que sejam feitos os ajustes necessários para a obtenção de bons resultados.

Com o surgimento de grandes modelos de linguagem (LLM) baseados em transformers~\cite{vaswani2017attention,devlin2019bert}, é possível observar um grande salto em termos de qualidade das ferramentas e métodos, o que pode ser atribuído à maior capacidade de mapeamento de conceitos em um espaço latente de atributos, que, por sua vez, possibilita um melhor agrupamento de textos em termos de semântica.

% A área de modelagem de tópicos é uma subárea de processamento de linguagem natural (NLP), que visa desenvolver métodos para extrair tópicos de documentos de texto e agrupar documentos tratando de assuntos semelhantes...\\
% Dada uma representação de uma série de documentos em um espaço latente, e uma categorização dos tópicos destes documentos, podemos analisar como estão distribuídos esses tópicos, e como eles se conectam, usando ferramentas teóricas de Teoria dos Grafos.

\section{Objetivos}

Neste trabalho, iremos investigar conexões entre tópicos de interesse de historiadores sobre a obra conhecida como "Etimologias", de Isidoro de Sevilha (c.560-636), que é uma compilação de 20 livros sobre as origens das palavras, em que o autor buscou registrar o conhecimento de escritores latinos da Antiguidade Clássica, como Varrão e Plínio o Velho. Esta obra é considerada a primeira grande enciclopédia da Idade Média, e foi copiada exaustivamente ao longo de cerca de 700 anos para ser utilizada como livro-texto base nas instituições de ensino da época.

O texto original é em latim medieval, mas para este trabalho será utilizada a tradução para a língua inglesa das Etimologias~\cite{barney2006etymologies}, por conveniência.

Para o estudo de conexões e tópicos existentes na obra, serão utilizadas ferramentas como grandes modelos de linguagem para geração de \emph{embeddings}, como Jina V3~\cite{jina}, bibliotecas de modelagem de tópicos baseadas em transformers, como BERTopic~\cite{grootendorst2022bertopic}, UMAP~\cite{mcinnes2018umap} para visualização dos \emph{embeddings} em duas dimensões, e Graphviz~\cite{gansner2000open} para visualização das conexões na forma de grafos.


\section{Plano de trabalho e cronograma}
As atividades a serem realizadas estão resumidas abaixo. Tanto a lista de atividades quanto o cronograma estão sujeitos a alterações.
\begin{enumerate}
    \item Conversas com especialistas para entender questões de interesse
    \item Estudo das ferramentas a serem utilizadas no projeto
    \item Preparação do texto para processamento
    \item Criação de ferramenta para modelagem de tópicos
    \item Análise dos resultados e validação com especialistas
    \item Apresentação do trabalho
    \item Produção do texto final

\end{enumerate}

\newcolumntype{B}{p{0.05\textwidth}}
\begin{table}[h]
    \centering
    \begin{tabular}{|c|c|c|c|c|c|c|c|c|}  
    \hline
    Atividades&Maio&Jun.&Jul.&Ago.&Set.&Out.&Nov.&Dez. \\
    \hline
    1&\cellcolor{black!100}& \cellcolor{black!100} & & & & & &\\
    \hline
    2& \cellcolor{black!100} & \cellcolor{black!100}&&&&&&\\
    \hline
    3&&\cellcolor{black!100}&\cellcolor{black!100}&&&&&\\
    \hline
    4&&&\cellcolor{black!100}&\cellcolor{black!100}&\cellcolor{black!100}&&&\\
    \hline
    5&&&&&\cellcolor{black!100}&\cellcolor{black!100}&&\\
    \hline
    6&&&&&&\cellcolor{black!100}&&\\
    \hline
    7&&&&&&\cellcolor{black!100}&\cellcolor{black!100}&\\
    \hline
    \end{tabular}
    \caption{Cronograma de execução mensal, de maio a dezembro de 2025}
    \label{table:Cronograma}
\end{table}

\bibliographystyle{alpha} 
\bibliography{dashboard}
\end{document}

