% Author: Nelson Lago
% This file is distributed under the MIT Licence

%%%%%%%%%%%%%%%%%%%%%%%%%%%%%%%%%%%%%%%%%%%%%%%%%%%%%%%%%%%%%%%%%%%%%%%%%%%%%%%%
%%%%%%%%%%%%%%%%%%%%%%%%%%%%%%%%% PREÂMBULO %%%%%%%%%%%%%%%%%%%%%%%%%%%%%%%%%%%%
%%%%%%%%%%%%%%%%%%%%%%%%%%%%%%%%%%%%%%%%%%%%%%%%%%%%%%%%%%%%%%%%%%%%%%%%%%%%%%%%

% A língua padrão é a última da lista
\documentclass[a1paper,brazilian,english]{article}

% Vários pacotes e opções de configuração genéricos
\usepackage{imegoodies}
\usepackage[poster,hidelinks, bluey]{imelooks}
\usepackage{wrapfig}
\usepackage{lipsum}
\tcbposterset{fontsize = 22pt} % default, mude se necessário

% Diretórios onde estão as figuras; com isso, não é necessário (mas
% é permitido) colocar o caminho completo em \includegraphics. Note
% que a extensão nunca é necessária (mas é permitida), ou seja, o
% resultado é o mesmo com "\includegraphics{figuras/foto.jpeg}",
% "\includegraphics{foto.jpeg}", "\includegraphics{figuras/foto}"
% ou "\includegraphics{foto}".
\graphicspath{{figuras/},{fig/},{logos/},{img/},{images/},{imagens/}}

% Comandos rápidos para mudar de língua:
% \en -> muda para o inglês
% \br -> muda para o português
% \texten{blah} -> o texto "blah" é em inglês
% \textbr{blah} -> o texto "blah" é em português
\babeltags{br = brazilian, en = english}


%%%%%%%%%%%%%%%%%%%%%%%%%%%%%%%%%%%%%%%%%%%%%%%%%%%%%%%%%%%%%%%%%%%%%%%%%%%%%%%%
%%%%%%%%%%%%%%%%%%%%%%%%%%%%%%%%%% METADADOS %%%%%%%%%%%%%%%%%%%%%%%%%%%%%%%%%%%
%%%%%%%%%%%%%%%%%%%%%%%%%%%%%%%%%%%%%%%%%%%%%%%%%%%%%%%%%%%%%%%%%%%%%%%%%%%%%%%%

% O arquivo com os dados bibliográficos para biblatex; você pode usar
% este comando mais de uma vez para acrescentar múltiplos arquivos
\addbibresource{bibliografia.bib}

% Este comando permite acrescentar itens à lista de referências sem incluir
% uma referência de fato no texto (pode ser usado em qualquer lugar do texto)
%\nocite{bronevetsky02,schmidt03:MSc, FSF:GNU-GPL, CORBA:spec, MenaChalco08}
% Com este comando, todos os itens do arquivo .bib são incluídos na lista
% de referências
%\nocite{*}


\begin{document}

% Em um poster não há \maketitle

\begin{tcbposter}[
  poster = {
    % showframe, % muito útil durante a preparação do poster
    rows = 6,
    columns = 12,
    colspacing = 1.2cm,
    rowspacing = .8cm,
  },
]

\posterbox[titlebox]{name=titlebox, below=top, column=1, span=12}{
  \begin{center}
    \huge{\textbf{Análise de Textos Históricos utilizando LLMs e Modelagem de Tópicos}}
  \end{center}
  % \begin{multicols}{2}
  %   \begin{center}
  %     Autor: João Pedro Lukasavicus Silva
  %   \end{center}

  %   \begin{center}
  %     Orientador: Mateus Espadoto
  %   \end{center}
  % \end{multicols}
}

\posterbox[footerbox]{name=footerbox, above=bottom, column=1, span=12}{
    linux.ime.usp.br/\textasciitilde jplukas/mac0499\par
    \small\ttfamily
    joao.lukasavicus.silva@usp.br\par
    \vspace{4pt}
    \footnotesize\rmfamily
    \textcolor{imesoftblue!30!white}
      {Departamento de Ciência da Computação --- Instituto de Matemática e Estatística, Universidade de São Paulo}
}


% \posterbox[adjusted title = The CCSL logo (full width)]
%           {name=widelogo, below=titlebox, column=1, span=12}{

%     \centering
%     \includegraphics[width=.95\textwidth]{ccsl-logo}
% }


%%%%%%%% Quatro colunas com "equal height group" %%%%%%%%

\posterbox[adjusted title = Introdução,
          smallmargins]
          {name=intro, below=titlebox, column=1, span=12}{

  Neste trabalho, iremos investigar conexões entre tópicos de interesse de historiadores
sobre a obra conhecida como "Etimologias", de Isidoro de Sevilha (c.560-636), que é uma
compilação de 20 livros sobre as origens das palavras, em que o autor buscou registrar o
conhecimento de escritores latinos da Antiguidade Clássica, como Varrão e Plínio o Velho.
Esta obra é considerada a primeira grande enciclopédia da Idade Média, e foi copiada
exaustivamente ao longo de cerca de 700 anos para ser utilizada como livro-texto base nas
instituições de ensino da época.

  % Para o estudo de conexões e tópicos existentes na obra, serão utilizadas ferramentas
  % como grandes modelos de linguagem para geração de embeddings, como Jina V3,
  % bibliotecas de modelagem de tópicos baseadas em transformers, como BERTopic,
  % UMAP [MHM18] para visualização dos embeddings em duas dimensões.

  % Tradicionalmente, pesquisadores das áreas de humanidades como história, filosofia, entre outras, dependem da análise de grandes quantidades de fontes para a realização do seu trabalho. Esta análise tipicamente é realizada de forma manual, e com grande custo, tanto em termos de tempo quanto de recursos humanos, custo este que pode se tornar um fator limitador da produção científica desses pesquisadores.

  % A área de processamento de linguagem natural (NLP) possui métodos bem estabelecidos para análise semântica de textos, como por exemplo, modelagem de tópicos~\parencite{deerwester1990indexing,blei2003latent,jelodar2019latent}, que visa agrupar textos que tratam de assuntos semelhantes. Com o surgimento dos grandes modelos de linguagem (LLM) baseados em transformers~\parencite{vaswani2017attention,devlin2019bert}, é possível observar um grande salto em termos de qualidade das ferramentas e métodos para análise de texto, o que pode ser atribuído à maior capacidade de mapeamento de conceitos em um espaço latente de atributos, que, por sua vez, possibilita um melhor agrupamento de textos em termos de semântica. Em todo caso, o uso destas técnicas requer certa familiaridade com o seu funcionamento e com os textos analisados, de modo que sejam feitos os ajustes necessários para a obtenção de bons resultados.
}

\posterbox[adjusted title = Modelagem de tópicos, smallmargins]{name=bertopic, below=intro, column=1, span=6}{

  \begin{wrapfigure}{r}{.3\textwidth}
    \includegraphics[width=\linewidth]{bertopic}
    \caption{Representação modular do BERTopic}
  \end{wrapfigure}

  Para a modelagem dos tópicos, utilizamos o BERTopic. Esta é uma técnica de modelagem de tópicos em linguagem natural que baseia-se (basicamente) em 4 passos:
  \begin{enumerate}
    \item Extração de representações semânticas vetoriais de sentenças (\textit{text embeddings});
    \item Redução de dimensionalidade;
    \item \textit{Clustering} das projeções de baixa dimensionalidade dos \textit{embeddings};
    \item Extração de representações textuais dos \textit{clusters}.
  \end{enumerate}
}

% \posterbox[adjusted title = 1. Mapeamento semântico,
%           smallmargins]
%           {name=sem_map, below=intro, column=4, span=3}{

%     Modelos de \textit{word embedding} e \textit{sentence embedding} são capazes de extrair informação semântica de palavras e frases em linguagem natural por meio de representações vetoriais...
% }

\posterbox[adjusted title = 1. Mapeamento semântico,
          smallmargins]
          {name=sem_map, below=bertopic, column=1, span=6}{

    Modelos de redes neurais artificiais denominados \textit{word} e \textit{sentence embedders} são capazes de extrair informação semântica de palavras e frases em linguagem natural, criando representações vetoriais.
    \begin{figure}[H]
      \begin{subfigure}[c]{.40\textwidth}
        \includegraphics[width=\textwidth]{embedding}
      \end{subfigure}
      \begin{subfigure}[c]{.40\textwidth}
        \includegraphics[width=\textwidth]{semantics}
      \end{subfigure}
      \caption{Exemplos de embeddings}
    \end{figure}
    Essas representações são construídas de tal maneira que sentenças (ou palavras) com significados análogos têm suas representações vetoriais próximas em um espaço latente de \textit{embeddings}.
}

\posterbox[adjusted title = 2. Redução de dimensionalidade: UMAP,smallmargins]{name=umap, below=sem_map, column=1, span=6}{

  Os modelos de embedding tipicamente projetam sentenças em espaços de alta dimensionalidade (\(d = 1000\), tipicamente). Devido à \textbf{maldição da dimensionalidade}, aplicamos uma técnica de redução de dimensionalidade denominada UMAP, para viabilizar a etapa posterior de \textit{clustering} (e também para visualização).\\
  \begin{figure}[H]
    \begin{subfigure}[c]{.40\textwidth}
      \includegraphics[width=\textwidth]{mammoth_3d}
      \caption{Original em 3D}
    \end{subfigure}
    \begin{subfigure}[c]{.40\textwidth}
      \includegraphics[width=\textwidth]{mammoth_2d}
      \caption{Representação em 2D, reduzida}
    \end{subfigure}
      \caption{Exemplo de visualização do UMAP}
  \end{figure}
}

\posterbox[adjusted title = 3. \textit{Clustering}: HDBSCAN, smallmargins]{name=hdbscan, below=umap, column=1, span=6}{
  Para a etapa de aglomeração dos documentos, utilizamos o HDBSCAN, que é um algoritmo de \textit{clustering} hierárquico baseado em densidade. Utilizamos essa abordagem para sermos capazes de detectar \textit{clusters} de formatos variados, não-convexos.
  \begin{figure}[H]
    \includegraphics[width=.40\textwidth]{hdbscan}
    \caption{Exemplo de clusterização pelo HDBSCAN. Observe os clusters (coloridos) de diferentes formas e densidades. Observe também a presença de pontos classificados como ruído}
  \end{figure}
}

% \posterbox[adjusted title = 4. Representação,smallmargins]{name=representacao, below=hdbscan, column=1, span=6}{

%     Escrever sobre vetorização, cTF-IDF, etc...
% }

\posterbox[adjusted title = Resultados, smallmargins]{name=resultados, below=intro, column=7, span=6}{
  \centering
  \includegraphics[width=\textwidth]{jina_embeddings}
}

\posterbox[adjusted title = Bibliografia,smallmargins]{name=bibliografia, between=resultados and footerbox, column=7, span=6}{

    bibliografia
}


\end{tcbposter}

\end{document}
