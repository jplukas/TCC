% Author: Nelson Lago
% This file is distributed under the MIT Licence

%%%%%%%%%%%%%%%%%%%%%%%%%%%%%%%%%%%%%%%%%%%%%%%%%%%%%%%%%%%%%%%%%%%%%%%%%%%%%%%%
%%%%%%%%%%%%%%%%%%%%%%%%%%%%%%%%% PREÂMBULO %%%%%%%%%%%%%%%%%%%%%%%%%%%%%%%%%%%%
%%%%%%%%%%%%%%%%%%%%%%%%%%%%%%%%%%%%%%%%%%%%%%%%%%%%%%%%%%%%%%%%%%%%%%%%%%%%%%%%

% A língua padrão é a última da lista
\documentclass[a1paper,brazilian,english]{article}

% Vários pacotes e opções de configuração genéricos
\usepackage{imegoodies}
\usepackage[poster,hidelinks, bluey]{imelooks}
% \tcbposterset{fontsize = 32pt} % default, mude se necessário

% Diretórios onde estão as figuras; com isso, não é necessário (mas
% é permitido) colocar o caminho completo em \includegraphics. Note
% que a extensão nunca é necessária (mas é permitida), ou seja, o
% resultado é o mesmo com "\includegraphics{figuras/foto.jpeg}",
% "\includegraphics{foto.jpeg}", "\includegraphics{figuras/foto}"
% ou "\includegraphics{foto}".
\graphicspath{{figuras/},{fig/},{logos/},{img/},{images/},{imagens/}}

% Comandos rápidos para mudar de língua:
% \en -> muda para o inglês
% \br -> muda para o português
% \texten{blah} -> o texto "blah" é em inglês
% \textbr{blah} -> o texto "blah" é em português
\babeltags{br = brazilian, en = english}


%%%%%%%%%%%%%%%%%%%%%%%%%%%%%%%%%%%%%%%%%%%%%%%%%%%%%%%%%%%%%%%%%%%%%%%%%%%%%%%%
%%%%%%%%%%%%%%%%%%%%%%%%%%%%%%%%%% METADADOS %%%%%%%%%%%%%%%%%%%%%%%%%%%%%%%%%%%
%%%%%%%%%%%%%%%%%%%%%%%%%%%%%%%%%%%%%%%%%%%%%%%%%%%%%%%%%%%%%%%%%%%%%%%%%%%%%%%%

% O arquivo com os dados bibliográficos para biblatex; você pode usar
% este comando mais de uma vez para acrescentar múltiplos arquivos
\addbibresource{bibliografia.bib}

% Este comando permite acrescentar itens à lista de referências sem incluir
% uma referência de fato no texto (pode ser usado em qualquer lugar do texto)
%\nocite{bronevetsky02,schmidt03:MSc, FSF:GNU-GPL, CORBA:spec, MenaChalco08}
% Com este comando, todos os itens do arquivo .bib são incluídos na lista
% de referências
%\nocite{*}


\begin{document}

% Em um poster não há \maketitle

\begin{tcbposter}[
  poster = {
    % showframe, % muito útil durante a preparação do poster
    rows = 6,
    columns = 11,
    colspacing = 1.2cm,
    rowspacing = .8cm,
  },
]

\posterbox[titlebox]{name=titlebox, below=top, column=1, span=11}{
  \begin{center}
    \huge{\textbf{Análise de Textos Históricos utilizando LLMs e Modelagem de Tópicos}}
  \end{center}
  % \begin{multicols}{2}
  %   \begin{center}
  %     Autor: João Pedro Lukasavicus Silva
  %   \end{center}

  %   \begin{center}
  %     Orientador: Mateus Espadoto
  %   \end{center}
  % \end{multicols}
}

\posterbox[footerbox]{name=footerbox, above=bottom, column=1, span=11}{
    linux.ime.usp.br/\textasciitilde jplukas/mac0499\par
    \small\ttfamily
    joao.lukasavicus.silva@usp.br\par
    \vspace{4pt}
    \footnotesize\rmfamily
    \textcolor{imesoftblue!30!white}
      {Departamento de Ciência da Computação --- Instituto de Matemática e Estatística, Universidade de São Paulo}
}


% \posterbox[adjusted title = The CCSL logo (full width)]
%           {name=widelogo, below=titlebox, column=1, span=12}{

%     \centering
%     \includegraphics[width=.95\textwidth]{ccsl-logo}
% }


%%%%%%%% Quatro colunas com "equal height group" %%%%%%%%

\posterbox[adjusted title = Introdução,
          smallmargins]
          {name=intro, below=titlebox, column=1, span=11}{

  Neste trabalho, iremos investigar conexões entre tópicos de interesse de historiadores
sobre a obra conhecida como "Etimologias", de Isidoro de Sevilha (c.560-636), que é uma
compilação de 20 livros sobre as origens das palavras, em que o autor buscou registrar o
conhecimento de escritores latinos da Antiguidade Clássica, como Varrão e Plínio o Velho.
Esta obra é considerada a primeira grande enciclopédia da Idade Média, e foi copiada
exaustivamente ao longo de cerca de 700 anos para ser utilizada como livro-texto base nas
instituições de ensino da época.

  % Para o estudo de conexões e tópicos existentes na obra, serão utilizadas ferramentas
  % como grandes modelos de linguagem para geração de embeddings, como Jina V3,
  % bibliotecas de modelagem de tópicos baseadas em transformers, como BERTopic,
  % UMAP [MHM18] para visualização dos embeddings em duas dimensões.

  % Tradicionalmente, pesquisadores das áreas de humanidades como história, filosofia, entre outras, dependem da análise de grandes quantidades de fontes para a realização do seu trabalho. Esta análise tipicamente é realizada de forma manual, e com grande custo, tanto em termos de tempo quanto de recursos humanos, custo este que pode se tornar um fator limitador da produção científica desses pesquisadores.

  % A área de processamento de linguagem natural (NLP) possui métodos bem estabelecidos para análise semântica de textos, como por exemplo, modelagem de tópicos~\parencite{deerwester1990indexing,blei2003latent,jelodar2019latent}, que visa agrupar textos que tratam de assuntos semelhantes. Com o surgimento dos grandes modelos de linguagem (LLM) baseados em transformers~\parencite{vaswani2017attention,devlin2019bert}, é possível observar um grande salto em termos de qualidade das ferramentas e métodos para análise de texto, o que pode ser atribuído à maior capacidade de mapeamento de conceitos em um espaço latente de atributos, que, por sua vez, possibilita um melhor agrupamento de textos em termos de semântica. Em todo caso, o uso destas técnicas requer certa familiaridade com o seu funcionamento e com os textos analisados, de modo que sejam feitos os ajustes necessários para a obtenção de bons resultados.
}

\posterbox[adjusted title = Modelagem de tópicos,
          smallmargins, equal height group=primer]
          {name=bertopic, below=intro, column=1, span=3}{

  Para a modelagem dos tópicos, utilizamos o BERTopic...
  \centering
  \includegraphics[width=\textwidth]{bertopic}
}

\posterbox[adjusted title = Mapeamento semântico,
          smallmargins, equal height group=primer]
          {name=sem_map, below=intro, column=4, span=3}{

    Modelos de \textit{word embedding} e \textit{sentence embedding} são capazes de extrair informação semântica de palavras e frases em linguagem natural por meio de representações vetoriais...
}

\posterbox[adjusted title = Redução de dimensionalidade,
          smallmargins]
          {name=umap, below=bertopic, column=1, span=3}{

    Escrever sobre redução de dimensionalidade e o UMAP
}

\posterbox[adjusted title = \textit{Clustering},
          smallmargins]
          {name=hdbscan, below=bertopic, column=4, span=3}{

    Escrever sobre clustering e o HDBSCAN
}

\posterbox[adjusted title = Resultados, smallmargins]{name=resultados, below=intro, column=7, span=5}{
  \centering
  \includegraphics[width=\textwidth]{jina_embeddings}
}

\posterbox[adjusted title = Bibliografia,
          smallmargins]
          {name=bibliografia, between=resultados and footerbox, column=7, span=5}{

    bibliografia
}





\end{tcbposter}

\end{document}
