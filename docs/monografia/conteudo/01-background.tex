%!TeX root=../monografia.tex
%("dica" para o editor de texto: este arquivo é parte de um documento maior)
% para saber mais: https://tex.stackexchange.com/q/78101

\chapter{Background}
\label{cap:background}
% Cap 1 (10 páginas): Background/Revisão bibliográfica. Nessa seção a ideia é contar um pouco do por quê do trabalho, do Isidoro, e das ferramentas (LLM, indexação semântica, Redução de Dimensionalidade, banco de dados vetorial, etc) que estamos usando em um nível mais genérico.

Neste trabalho, utilizaremos ferramentas de modelagem de tópicos e grandes modelos de linguagem com o objetivo de demonstrar sua utilidade no estudo de textos históricos. Nas seções a seguir detalhamos as ferramentas utilizadas e as obras a serem estudadas.

\section{Redes Neurais e Grandes Modelos de Linguagem}

Desde o surgimento dos primeiros computadores digitais no período logo após a Segunda Guerra Mundial (1939 - 1945), pesquisadores trabalharam para desenvolver ferramentas que pudessem executar tarefas que até então eram realizadas apenas por humanos. Cientistas como Warren McCulloch (1898 - 1969) e Walter Pitts (1923 - 1969), um médico e um lógico, estão entre os primeiros a discutirem modelos do cérebro humano, e como estes poderiam ser implementados computacionalmente. Em 1943 publicaram um artigo~\parencite{McCulloch1943} considerado seminal na área, aonde propuseram um modelo matemático do cérebro representado como uma rede de elementos simples interconectados. Um elemento particular recebe sinais dos elementos conectados à sua entrada, e produzem uma soma ponderada como sinal de saída, que, dependendo de um limiar pré-determinado, é enviado para elementos conectados à sua saída como um sinal ativo (valor 1) ou inativo (valor 0). Os autores demonstram no artigo que é possível calcular funções lógicas utilizando diferentes configurações de conexões entre elementos. Posteriormente, estes elementos passaram a ser conhecidos como neurônios artificiais, que serviram de base para o desenvolvimento das redes neurais artificiais, anos mais tarde.

Alan Turing (1912 - 1954), matemático e teórico da computação que ajudou a estabelecer as bases da ciência da computação como disciplina, foi um dos primeiros a se dedicar ao que veio a ser chamado posteriormente de inteligência artificial, de forma mais abrangente. Turing apresenta~\parencite{alan1950a} um conjunto de questões e definições que seriam essenciais para o desenvolvimento da área. Em primeiro lugar, ele pergunta se “máquinas podem pensar”, o que traz o primeiro problema importante, o de falta de definições claras sobre o que é ``pensar'' e, naquele momento, sobre o que seria a ``máquina'' em questão. Como definições do que é pensar, e de forma mais abrangente, do que é inteligência, representam problemas filosóficos importantes e sem resposta objetiva, Turing propôs uma abordagem prática para isso: no lugar de ``máquinas podem pensar'', ele passa a perguntar se ``máquinas podem agir de forma indistinguível de um humano''. Em outras palavras, ele ignora o processo interno pelo qual uma ação é produzida, e se preocupa apenas com o efeito da ação por aqueles que a percebem. Para demonstrar esta ideia, Turing propôs um experimento chamado de Jogo da Imitação. Na versão mais simples do jogo, há três participantes: um humano e um computador que devem responder a perguntas de um juiz humano que só pode fazer perguntas e receber respostas escritas em papel, sem contato direto com os outros dois. Se o juiz, após uma série de perguntas, não for capaz de distinguir o humano da máquina, a máquina venceu o jogo, ou seja, conseguiu se passar por um humano.
Apesar de ser uma definição prática, simples, e que serve aos propósitos do autor, esta ideia de que basta ser percebido como humano para ser considerado inteligente ou pensante é problemática para pensadores de outras áreas, que tratam de aspectos metafísicos da mente e do pensamento. 

Em 1980, o filósofo John Searle (1932 -) propôs o argumento do Quarto Chinês~\parencite{searle1980minds}. Neste experimento, o autor imagina uma pessoa que não fala chinês isolada em um quarto com um livro contendo instruções sobre como interpretar símbolos chineses. Se alguém colocar um texto em chinês por debaixo da porta, por exemplo, a pessoa poderia seguir as instruções do livro para produzir símbolos que representem uma resposta coerente para falantes de chinês. No entanto, esta pessoa estaria apenas seguindo regras sintáticas sem nenhuma compreensão semântica do texto que está produzindo. O autor busca com isso se contrapor a ideias funcionalistas, como a de Turing, de que a mente é apenas um sistema de processamento de informações. Ou seja, para Searle, não basta ser capaz de se comportar como se entendesse uma conversa para uma entidade ser considerada como pensante ou inteligente.

Apesar das críticas, a busca pela inteligência artificial seguiu com maior ou menor intensidade ao longo da segunda metade do século XX. Por exemplo, partindo da ideia de neurônios artificiais de McCulloch e Pitts em 1943, Frank Rosenblatt (1928 - 1971), psicólogo, desenvolveu o Perceptron~\parencite{rosenblatt1958perceptron} em 1957, que é considerada a primeira rede neural artificial. A ideia do Perceptron foi desenvolvida ao longo dos anos, passando por diversos altos e baixos, e com a evolução do hardware existente, cada vez mais capaz, culminou na criação do que passou a ser chamado de Deep Learning~\parencite{alexnet}, por volta de 2012 com o trabalho de Alex Krizhevsky (1986 -), que consiste no uso de grandes redes neurais com milhões e até bilhões de elementos aplicado a problemas de processamento de texto e imagem. Mais recentemente, por volta de 2017, a partir do trabalho de Ashish Vaswani (1986 -) e outros foram criadas novas arquiteturas de conexão entre os neurônios artificiais, batizadas de Transformers~\parencite{vaswani2017attention}, com o objetivo de permitir que mais dados pudessem ser armazenados e processados em conjunto, desta forma possibilitando fornecer mais informações de contexto para um problema computacional de processamento de texto ou imagem. A ideia dos transformers é que possibilitou o surgimento de modelos mais recentes como ChatGPT e Gemini, que são exemplares do que é considerado inteligência artificial hoje em dia. 

Modelos baseados em transformers possuem grande capacidade de mapeamento semântico de dados e de geração de texto verossímil, que podem ser consideradas como suas características mais salientes. Estas capacidades são obtidas com base na observação e processamento de grande quantidade de dados de exemplo, processo que é chamado de ``treinamento'' no jargão da área, realizados com o objetivo de se identificar padrões existentes nos dados para poder determinar, por exemplo, quais palavras que aparecem comumente próximas a outras em certos contextos. Concluída esta etapa de treinamento, o modelo é capaz de produzir texto em resposta a uma questão: primeiro é feito o mapeamento semântico da questão para encontrar termos e sentenças com significado parecido na memória do modelo, e segundo, com base nos termos e sentenças encontrados, o modelo faz a síntese de um texto que tenha verossimilhança. Note o uso do termo verossimilhança no lugar de corretude: como os modelos são criados de acordo com a definição de Turing, basta que forneçam respostas indistinguíveis das de um humano, e cabe a quem coloca a questão avaliar a corretude das respostas fornecidas. 

TODO: escrever +- 0.5 pagina sobre como transformers funcionam, em termos gerais. É bom ter algumas figuras.

\section{Modelagem de Tópicos}

TODO: escrever +- 0.5 pagina sobre modelagem de tópicos. É bom ter algumas figuras.

\section{Obras e autores estudados}

Neste trabalho utilizaremos edições em inglês das seguintes obras: as Etimologias (\emph{Etymologiae}), de Isidoro de Sevilha, Sobre Agricultura (\emph{De Agri Cultura}), de Catão, e Sobre Ciência Agrícola (\emph{De Re Rustica}), de Varrão. A escolha destas obras não é por acaso: estudiosos da obra de Isidoro consideram que ele se baseou em diversas obras da antiguidade para escrever as Etimologias, particularmente as obras de Catão e Varrão sobre agricultura. Sendo assim, tentaremos encontrar similaridades de ideas entre os textos a partir do uso de modelos computacionais. A seguir apresentaremos detalhes sobre os autores e seu contexto.

Isidoro de Sevilha (c. 560 - 636) foi um clérigo, teólogo e pensador da alta idade média, que é considerado um dos intelectuais mais importantes do seu tempo, e cuja influência foi sentida por muitos séculos após a sua morte. Nasceu em Cartagena, que à época era parte do Reino Visigótico, estado que ocupou as regiões da  Península Ibérica e atual sul da França no período seguinte ao fim do Império Romano do ocidente, e fazia parte de uma família que percentia à elite hispano-romana. Seus três irmãos ocuparam funções importantes na igreja, com destaque para seu irmão mais velho, Leandro de Sevilha (c. 534 - c. 600) foi Bispo de Sevilha, cargo que Isidoro ocuparia após a morte de Leandro. Todos os quatro irmãos são venerados como santos pela Igreja Católica.

Como Bispo de Sevilha, Isidoro exerceu grande influência no seu tempo, presidindo sínodos e concílios importantes, como os de Sevilha e Toledo, protegendo os monastérios, e ainda se envolveu na conversão dos reis Visigodos do Arianismo, uma doutrina cristã não-trinitária, para o cristianismo Calcedoniano, que veio a se tornar a doutrina dominante na Igreja Católica.

Como intelectual, produziu diversas obras, dentre as quais se destacam as Etimologias, que são um conjunto de livros que formam uma enciclopédia etimológica, que resume e organiza o conhecimento de diversos autores da antiguidade clássica. A obra segue a tradição de enciclopedistas clássicos, como Plínio, o Velho (c. 23 - 79), com o uso de ordenação alfabética de tópicos e de uma abordagem literária para o conhecimento, baseada no pensamento analógico.

As Etimologias tratam de temas diversos como gramática, retórica, matemática, direito, a Igreja, heresias, guerra, agricultura, entre outros. A sua influência foi tão grande nos séculos seguintes que algumas das obras clássicas utilizadas como base deixaram de ser lidas e copiadas e acabaram se perdendo no tempo. Era considerado o texto base para a educação sobre o período clássico durante a maior parte da idade média.

O estudo crítico das Etimologias revela suas possíveis fontes clássicas, que na maioria das vezes não são citadas por Isidoro. De acordo com Stephen A. Barney, tradutor para o inglês~\parencite{barney2006etymologies} das Etimologias, é possível identificar que o material dos livros I e II, que tratam de gramática, retórica e dialética (as disciplinas do \emph{trivium}) provavelmente foram extraídos dos Institutos, de Cassiodoro (c. 485 - c. 585), o livro III, sobre matemática (contendo as disciplinas do \emph{quadrivium}), provavelmente foi inspirado em Boécio (c. 480 - 524), e o livro XVII, sobre agricultura, deriva de Catão, o Velho (234 - 149 a.C.) e Varrão (116 - 27 a.C.), para citar alguns exemplos.

Marco Pórcio Catão, o Velho (234 - 149 a.C.), foi um soldado, senador e historiador romano. Nascido em uma família de plebeus, que era a classe baixa de cidadãos livres em Roma, desecndia de gerações de soldados com reputação de bravura, como seu pai e seu bisavô. Ainda na infância, com a morte de seu pai, passou a cuidar das atividades da fazenda família. Como jovem soldado, especula-se que aos 20 anos tenha participado de campanhas das Guerras Púnicas no papel de tribuno militar, uma patente de oficial do exército romano. Ao retornar do campo de batalha para a sua fazenda, e com o apoio do seu vizinho e amigo Lúcio Valério Flaco, iniciou carreira política como \emph{questor}, cargo que possuía diversas atribuições, entre elas a de cobrança de impostos e de supervisão financeira. Daí se seguiram diversos cargos políticos importantes, como \emph{pretor} (magistrado), \emph{consul} (o cargo mais alto da República Romana) e \emph{censor} (magistrado de nível superior). Em paralelo a suas atividades políticas, escreveu diversas obras, cuja maioria infelizmente foi perdida. Escreveu uma história de Roma em sete livros chamada de Origens, uma obra sobre assuntos militares, e a obra sobre agricultura que utilizaremos neste trabalho, a única preservada na íntegra. Além disso, foi famoso orador e teve cerca de 150 discursos registrados.

Marco Terêncio Varrão (116 - 27 a.C.) foi um intelectual e polímata romano, descrito por Petrarca como a ``terceira grande luz'' de Roma, depois de Virgílio e Cícero. Nascido em família pertencente à classe equestre de Roma, abaixo somente da classe senatorial, ocupou cargos políticos ao longo da vida, como \emph{questor}, \emph{pretor} e tribuno do povo. Estudou com o filólogo romano Lúcio Élio Estilo e com o filósofo platonista Antíoco de Ascalão. Foi também um líder militar sem grande prestígio durante a Guerra Civil Cesariana. Foi um escritor prolífico, que produziu cerca de 74 obras sobre temas diversos, entre as quais se destacam a Cronologia Varroniana, que lista as datas de eventos importantes de Roma, e os nove livros das Disciplinas, organizados de acordo com os temas das artes liberais da época, e que serviram de exemplo para enciclopedistas que vieram posteriormente, como Plínio, o Velho, e o próprio Isidoro de Sevilha.

