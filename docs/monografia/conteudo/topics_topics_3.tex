\subsubsection{Tópico 3: \textit{sun, sky, stars, air, day, moon, earth, constellations, star, month}}
\label{tema:topics_3}
\begin{description}
\footnotesize
\item [Livro: 3 - Mathematics, music, astronomy (58.46\%)]\hfill
\begin{itemize}
\item[\textbf{isi\_en.3.L.2.Sall}]
\textit{2. Wandering farther to the south it makes winter, so that the earth grows fertile with wintry moisture and frost. When it approaches closer to the north, it brings summer back, so that crops grow firm in ripeness, and what was unripened in damp weather mellows in its warmth. }
\item[\textbf{isi\_en.3.LI.2.Sall}]
\textit{2. When the sun runs across the south, it is the closer to the earth; but when it is near the north, it is raised higher in the sky. [Thus God made diverse locations and seasons for the sun's course, so that it does not consume everything with its daily heat by always tarrying in the same place. But, as Clement said, "The sun takes diverse paths, by means of which the temperature of the air is meted out according to the pattern of the seasons, and the order of its changes and permutations is preserved. Thus when the sun ascends to the higher reaches, it tempers the spring air; when it reaches its zenith, it kindles the summer heat; dropping again it brings back the temperance of autumn. But when it goes back to the lowest orbit, it bequeaths to us from the icy framework of the sky the rigor of winter cold."]  }
\item[\textbf{isi\_en.3.LIII.2.Sall}]
\textit{2. Others maintain on the contrary that the moon does not have its own light, but is illuminated by the rays of the sun, and for this reason undergoes an eclipse when the earth's shadow comes between it and the sun. [For the sun is located higher than the moon. Hence it happens that when the moon is beneath the sun, the upper part of the moon is lighted, but the lower part, which is facing the earth, is dark.] }
\end{itemize}
\item [Livro: 13 - The cosmos and its parts (53.44\%)]\hfill
\begin{itemize}
\item[\textbf{isi\_en.13.I.3.Sall}]
\textit{3. There are four zones in the world, that is, four regions: the East and the West, the North and the South. }
\item[\textbf{isi\_en.13.II.4.Sall}]
\textit{4. In number, take for example eight divided into four, and four into two, and then two into one. But one is an atom, because it is indivisible. Thus also with letters (i.e. speech-sounds), for speech is divided into words, words into syllables, syllables into letters. But a letter, the smallest part, is an atom and cannot be divided. Therefore an atom is whatever cannot be divided, like a point in geometry, for tóµoç means "division" in Greek, and ?toµoç means "non-division." }
\item[\textbf{isi\_en.13.III.3.Sall}]
\textit{3. For this reason, all the elements are present in all, but each one has taken its name from whichever element is more abundant in it. The elements are assigned by Divine Providence to the appropriate living beings, for the Creator himself has filled heaven (i.e. the fiery realm) with angels, air with birds, water with fish, and earth with humans and the rest of the living things. }
\end{itemize}
\item [Livro: 5 - Laws and times (33.15\%)]\hfill
\begin{itemize}
\item[\textbf{isi\_en.5.XXIX.1.Sall}]
\textit{1. Intervals of time are divided into moments, hours, days, months, years, lustrums, centuries, and ages. A moment (momentum) is the least and shortest bit of time, so called from the movement (motus) of the stars. }
\item[\textbf{isi\_en.5.XXVII.38.Sall}]
\textit{38. For that reason the Romans forbade water and fire to certain condemned people - because air and water are free to all and given to everyone - so that the condemned might not enjoy what is given by nature to everyone. }
\item[\textbf{isi\_en.5.XXX.16.Sall}]
\textit{16. Evening (suprema) is the last part of the day, when the sun turns its course toward its setting - so called because it 'still exists' (superesse) up to the final part of the day. }
\end{itemize}
\item [Livro: 8 - The Church and sects (28.97\%)]\hfill
\begin{itemize}
\item[\textbf{isi\_en.8.IX.17.Sall}]
\textit{17. Haruspices are so named as if the expression were 'observers (inspector) of the hours (hora)'; they watch over the days and hours for doing business and other works, and they attend to what a person ought to watch out for at any particular time. They also examine the entrails of animals and predict the future from them. }
\item[\textbf{isi\_en.8.VI.21.Sall}]
\textit{21. Whence also Varro says that fire is the soul of the world; just as fire governs all things in the world, so the soul governs all things in us. As he says most vainly, "When it is in us, we exist; when it leaves us, we perish." Thus also when fire departs from the world through lightning, the world perishes. }
\item[\textbf{isi\_en.8.XI.45.Sall}]
\textit{45. Mercury (Mercurius) is translated as "speech," for Mercury is said to be named as if the word were mediuscurrens ("go-between"), because speech is the gobetween for people. In Greek he is called `Epµ?ç, because 'speech' or 'interpretation,' which pertains especially to speech, is called spµ?v?(c)a. }
\end{itemize}
\item [Livro: 6 - Books and ecclesiastical offices (8.91\%)]\hfill
\begin{itemize}
\item[\textbf{isi\_en.6.II.3.Sall}]
\textit{3. The book of Genesis is so called because the beginning of the world and the begetting (generatio) of living creatures are contained in it. }
\item[\textbf{isi\_en.6.XIX.2.Sall}]
\textit{2. The office of Vespers takes place at the beginning of night, and is named for the evening star Vesper, which rises when night falls. }
\item[\textbf{isi\_en.6.XVII.3.Sall}]
\textit{3. It is calleda cycle (cyclum) because it is set out in the form of a wheel, and arranged as if it were in a circle (circulum) it comprises the order of the years without variation and without any artifice. }
\end{itemize}
\end{description}
