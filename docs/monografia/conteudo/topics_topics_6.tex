\subsubsection{Tópico 6: \textit{stone, color, bronze, white, iron, gold, red, black, lead, silver}}
\label{tema:topics_6}
\begin{description}
\footnotesize
\item [Livro: 16 - Stones and metals (84.92\%)]\hfill
\begin{itemize}
\item[\textbf{isi\_en.16.I.10.Sall}]
\textit{10. There are four kinds of sulfur. The 'living' kind, which is dug up, is translucent and green; physicians use it alone of all the kinds of sulfur. The second kind, which people call 'lump sulfur' (i.e. fuller's earth), is used only by fullers. The third kind is 'liquid'; it is used for fumigating wool because it gives whiteness and softness. The fourth kind is particularly suitable for preparing lamp-wicks. The power of sulfur is so great that it cures the 'comitial sickness' (see IV.vii.7) through its vapor when it is set on the fire and burns. When a person puts sulfur in a goblet of wine and carries it around with hot coals beneath he glows with the eerie pallor of a corpse from the reflection of the blaze. }
\item[\textbf{isi\_en.16.II.10.Sall}]
\textit{10. But it is now produced elsewhere in caves, because, having collected as a liquid, it drips down there and solidifies into 'grape clusters.' It also occurs in hollow trenches, from whose sides the hanging drops coalesce; it is also made, like salt, under the most blazing sun. Its power is so concentrated that, when sprinkled into the mouths of lions and bears, they are unable to bite because of its astringent force. }
\item[\textbf{isi\_en.16.III.11.Sall}]
\textit{11. Sand (arena, i.e. harena) is named from 'dryness' (ariditas), not from 'cementing' (adhaerere) building materials, as some people would have it. The test of its quality is if it grates when squeezed in one's hand, or if it leaves behind no stain when sprinkled on white cloth. }
\end{itemize}
\item [Livro: 19 - Ships, buildings, and clothing (18.00\%)]\hfill
\begin{itemize}
\item[\textbf{isi\_en.19.X.3.Sall}]
\textit{3. Stones that are suitable for building: white stone, Tiburtine, columbinus, river stone, porous, red, and the others. }
\item[\textbf{isi\_en.19.XVII.5.Sall}]
\textit{5. Syrian (Syricum) is a pigment with a red color, with which the chapter-heads in books are written. It is also known as Phoenician, so called because it is collected in Syria on the shores of the Red Sea, where the Phoenicians live. }
\item[\textbf{isi\_en.19.XXVIII.1.Sall}]
\textit{1. Dying (tinctura) is so named because cloth is 'soaked in color' (tinguere), tinted to another appearance, and colored for the sake of beauty. What we call red or vermilion (vermiculus), the Greeks call mómmoç; it is a small grub (vermiculus) from the foliage of the forest. }
\end{itemize}
\item [Livro: 13 - The cosmos and its parts (3.05\%)]\hfill
\begin{itemize}
\item[\textbf{isi\_en.13.X.1.Sall}]
\textit{1. The celestial rainbow (arcus) is named for its likeness to the curve of a bow (also arcus). Iris is its proper name. It is called iris as if the word were aeris, that is, something that descends to earth through the air (aer). It takes its light from the sun, whenever hollow clouds receive the sun's rays from the opposite side and make the shape of a bow. This circumstance gives it various colors, because the thinned water, bright air, and misty clouds, when illuminated, create various colors. }
\item[\textbf{isi\_en.13.XVII.2.Sall}]
\textit{2. The Red Sea is so named because it is colored with reddish waves; however, it does not possess this quality by its nature, but its currents are tainted and stained by the neighboring shores because all the land surrounding that sea is red and close to the color of blood. From there a very intense vermilion may be separated out, as well as other pigments with which the coloring of paintings is varied. }
\item[\textbf{isi\_en.13.XX.5.Sall}]
\textit{5.A drop (gutta) is that which stands hanging, stilla (i.e. another word for 'drop') is that which falls. Hence the word stillicidium (i.e. drippings from eaves), as if it were 'falling drop' (stilla cadens). Stiria (lit. "frozen drop, icicle," here simply "drop") is a Greek word, that is 'drop' (gutta); from it the diminutive that we call stilla is formed. As long as it stands or hangs suspended from roofs or trees, it is a gutta, as if 'glutinous' (glutinosus), but when it has fallen it is a stilla. }
\end{itemize}
\item [Livro: 20 - Provisions and various implements (3.05\%)]\hfill
\begin{itemize}
\item[\textbf{isi\_en.20.II.31.Sall}]
\textit{31. Galatica is named for its milky color, for the Greeks call milk y?2a. Meatballs (sphera) are named with this Greek word for their roundness, for whatever is round in shape is called o??a±pa in Greek. }
\item[\textbf{isi\_en.20.IV.8.Sall}]
\textit{8. Chrysendetus vessels are those inlaid with gold; the term is Greek (cf. ypUoóç, "gold"; sv6?±v "bind on"). Vessels in bas-relief (anaglypha) are those carved on top, for the Greek ?vY means "above," y2U???, "carving," that is, carved above. }
\item[\textbf{isi\_en.20.VII.2.Sall}]
\textit{2. An alabastrum isa vessel for ointments, and is named from the kind of stone of which it is made, which they call alabaster (alabastrites), which keeps ointments unspoiled. }
\end{itemize}
\item [Livro: 4 - Medicine (1.56\%)]\hfill
\begin{itemize}
\item[\textbf{isi\_en.4.VI.16.Sall}]
\textit{16. A carbuncle (carbunculus) is so called because at first it glows red, like fire, and then turns black, like an extinguished coal (carbo). }
\item[\textbf{isi\_en.4.VII.32.Sall}]
\textit{32. A cauculus is a stone that occurs in the bladder, and it took its name from that (i.e. calculus, "pebble"). It is formed from phlegmatic matter. }
\end{itemize}
\end{description}
