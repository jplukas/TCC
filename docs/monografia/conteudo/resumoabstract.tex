%!TeX root=../monografia.tex
%("dica" para o editor de texto: este arquivo é parte de um documento maior)
% para saber mais: https://tex.stackexchange.com/q/78101

% As palavras-chave são obrigatórias, em português e em inglês, e devem ser
% definidas antes do resumo/abstract. Acrescente quantas forem necessárias.
\palavraschave{Redes neurais, LLM, Grandes modelos de linguagem, Modelagem de tópicos, Textos históricos, Isidoro de Sevilha, Etimologias}

\keywords{Neural Networks, LLM, Large Language Models, Topic Modeling, Historical Texts, Isidore of Seville, Etymologies}

% O resumo é obrigatório, em português e inglês. Estes comandos também
% geram automaticamente a referência para o próprio documento, conforme
% as normas sugeridas da USP.
\resumo{
Este trabalho investiga a aplicação de Grandes Modelos de Linguagem (LLMs) e técnicas de modelagem de tópicos na análise de textos históricos, com foco nas Etimologias de Isidoro de Sevilha. Enquanto a análise tradicional de tais \emph{corpora} extensos é manual e custosa, este estudo propõe um \emph{pipeline} computacional para detectar e organizar automaticamente estruturas temáticas latentes. Foram realizados experimentos utilizando \emph{embeddings} de sentenças para avaliar a qualidade dos tópicos gerados. Os resultados indicam que uma configuração que prioriza a minimização do tamanho máximo dos clusters, em detrimento da minimização de outliers, produz grupos semanticamente mais coerentes. Além disso, a análise ao nível de sentenças mostrou-se superior no isolamento de assuntos distintos, como gramática e retórica, revelando também temas transversais dispersos pelos livros. Conclui-se que estes métodos computacionais fornecem uma ferramenta robusta de "leitura distante", capaz de replicar a categorização de especialistas e descobrir conexões semânticas na literatura medieval.
}

\abstract{
This work investigates the application of Large Language Models (LLMs) and topic modeling techniques to the analysis of historical texts, focusing on the Etymologies of Isidore of Seville. While traditional analysis of such extensive corpora is manual and labor-intensive, this study proposes a computational pipeline to automatically detect and organize latent thematic structures. Experiments were conducted using sentence embeddings to assess the quality of the generated topics. The results indicate that a configuration which prioritizes the minimization of maximum cluster size, rather than the minimization of outliers, yields more semantically coherent groups. Furthermore, sentence-level analysis proved superior in isolating distinct subjects, such as grammar and rhetoric, while also revealing transversal themes dispersed throughout the books. We conclude that these computational methods provide a robust "distant reading" tool, capable of replicating expert categorization and uncovering semantic connections in medieval literature.
}
