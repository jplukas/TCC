\subsubsection{Tópico 4: \textit{exile, ignarus, extra, comparative, comparison, learned, terror, doctus, outside, ignorant ignarus}}
\label{tema:sentences_4}
\begin{description}
\footnotesize
\item [Livro: 10 - Vocabulary (33.00\%)]\hfill
\begin{itemize}
\item[\textbf{isi\_en.10.A.8.S3}]
\textit{Ambitious (ambitiosus), because one solicits (ambire) honors.}
\item[\textbf{isi\_en.10.B.31.S0}]
\textit{Baburrus, "stupid, inept."}
\item[\textbf{isi\_en.10.C.61.S1}]
\textit{Similarly a lunatic (lunaticus) because [}
\item[\textbf{isi\_en.10.D.77.S3}]
\textit{Indolent (desidiosus), "sluggish, lazy," so called from 'settling down' (desidere), that is, from sitting too much.}
\item[\textbf{isi\_en.10.E.83.S3}]
\textit{Destitute (expers), because 'without a share (pars),' for such a one lacks a share.}
\end{itemize}
\item [Livro: 1 - Grammar (3.29\%)]\hfill
\begin{itemize}
\item[\textbf{isi\_en.1.I.1.S2}]
\textit{A discipline is so named in another way, because 'the full thing is learned' (discitur plena).}
\item[\textbf{isi\_en.1.VI.2.S2}]
\textit{The adverb is taken from the noun, as in 'a learned one, learnedly' (doctus, docte).}
\item[\textbf{isi\_en.1.VII.1.S1}]
\textit{Indeed, unless you know its name (nomen), the knowledge of a thing perishes.}
\item[\textbf{isi\_en.1.XXVII.15.S0}]
\textit{There is a question about how maxumus or maximus ("greatest"), and any similar pairs, ought to be written.}
\item[\textbf{isi\_en.1.XXVIII.2.S0}]
\textit{If any one of these is lacking, it is no longer analogy, that is, similarity, but rather anomaly, that is, outside the rule, such as lepus ("hare") and lupus ("wolf").}
\end{itemize}
\item [Livro: 5 - Laws and times (2.07\%)]\hfill
\begin{itemize}
\item[\textbf{isi\_en.5.XXVII.30.S0}]
\textit{Proscription (proscriptio) is a condemnation of exile at a distance, as if it were a 'writing afar' (porro scriptio).}
\end{itemize}
\item [Livro: 2 - Rhetoric and dialectic (1.84\%)]\hfill
\begin{itemize}
\item[\textbf{isi\_en.2.IX.15.S1}]
\textit{is] an enemy."}
\item[\textbf{isi\_en.2.V.7.S1}]
\textit{Comparison (comparatio) occurs when some deed of another person is argued to be proper and useful, because as that deed happened, so this deed at issue is said to have been committed.}
\item[\textbf{isi\_en.2.XXI.9.S1}]
\textit{Rutilius Lupus, Schemata Lexeos 1.4): "While you call yourself wise instead of cunning, brave instead of reckless, thrifty instead of stingy."}
\item[\textbf{isi\_en.2.XXIV.6.S2}]
\textit{Temperance (temperantia), how passion and the desire for things may be reined in.}
\end{itemize}
\item [Livro: 8 - The Church and sects (0.79\%)]\hfill
\begin{itemize}
\item[\textbf{isi\_en.8.V.6.S0}]
\textit{The Gnostics (Gnosticus) wish to call themselves thus because of the superiority of their knowledge (cf. yv?otç, "knowledge").}
\item[\textbf{isi\_en.8.XI.88.S0}]
\textit{They name Genius (Genius) thus because he possesses the force, as it were, of generating (gignere, ppl.}
\end{itemize}
\end{description}
