\subsubsection{Tópico 1: \textit{tree, stone, color, grows, white, bronze, black, fruit, leaves, root}}
\label{tema:sentences_1}
\begin{description}
\footnotesize
\item [Livro: 17 - Rural matters (84.50\%)]\hfill
\begin{itemize}
\item[\textbf{isi\_en.17.II.7.S0}]
\textit{7.A grain-field (seges) is [}
\item[\textbf{isi\_en.17.III.10.S4}]
\textit{The second is distichon barley, because it has two rows; many call this Galaticum.}
\item[\textbf{isi\_en.17.IV.11.S0}]
\textit{11. 'Bitter vetch' (ervum) derives its name from Greek, for they call it opoßoç, because while it is dangerous for some livestock it nevertheless makes bulls (cf. ßouç, "bull") plump.}
\item[\textbf{isi\_en.17.IX.106.S2}]
\textit{We call a handful of hay a 'sheaf' (manipulum), and it is called manipulum because it "fills the hand" (manum implere).}
\item[\textbf{isi\_en.17.V.3.S0}]
\textit{The labrusca is a wild vine that grows in marginal land, whence it is called labrusca, from the margins (labrum) and extremities of the land.}
\end{itemize}
\item [Livro: 16 - Stones and metals (72.24\%)]\hfill
\begin{itemize}
\item[\textbf{isi\_en.16.I.10.S2}]
\textit{The second kind, which people call 'lump sulfur' (i.e. fuller's earth), is used only by fullers.}
\item[\textbf{isi\_en.16.II.3.S6}]
\textit{In some places it is so hard that people make walls and houses out of masses of salt, as in Arabia.}
\item[\textbf{isi\_en.16.III.9.S0}]
\textit{Gypsum (gypsum) is related to limestone; it is a Greek term (i.e. yáyoç).}
\item[\textbf{isi\_en.16.IV.26.S0}]
\textit{The stone melanites is so called because it exudes a sweet honeyed (melleus) fluid.}
\item[\textbf{isi\_en.16.IX.5.S0}]
\textit{Amethystizontas is so named because the sparkle on its surface ranges towards the violet color of amethyst.}
\end{itemize}
\item [Livro: 20 - Provisions and various implements (40.89\%)]\hfill
\begin{itemize}
\item[\textbf{isi\_en.20.II.23.S0}]
\textit{Fried (frixus) is so termed from the sound food makes when it is seared in oil.}
\item[\textbf{isi\_en.20.III.10.S1}]
\textit{Honey-wine (mulsum) is wine mixed with honey, for it is a drink made from water and honey, which the Greeks call µ?2(c)mpatov.}
\item[\textbf{isi\_en.20.IV.3.S0}]
\textit{3. Ceramic dishes are said to have been first invented on the island of Samos, made from white clay and hardened by fire, hence 'Samian dishes.'}
\item[\textbf{isi\_en.20.V.4.S2}]
\textit{An amystis (cf. ?}
\item[\textbf{isi\_en.20.VI.2.S2}]
\textit{Later they passed into use for wine, but keeping the Greek term with which they had their origin.}
\end{itemize}
\item [Livro: 19 - Ships, buildings, and clothing (23.99\%)]\hfill
\begin{itemize}
\item[\textbf{isi\_en.19.II.9.S2}]
\textit{): The shining mast-heads (carchesium) of the tall mast glitter.}
\item[\textbf{isi\_en.19.IV.10.S0}]
\textit{The sounding-lead (catapirates) is a line with a lead weight, with which the depth of the sea is tested.}
\item[\textbf{isi\_en.19.VI.7.S4}]
\textit{Once extinguished it lasts so incorruptibly that the people who fix boundaries spread charcoals below the surface and place stones on top, so as to prove the boundary to a litigant however many generations later, and they recognize a stone fixed in this way to be a boundary.}
\item[\textbf{isi\_en.19.VII.1.S0}]
\textit{The anvil (incus) is that tool on which iron is beaten out.}
\item[\textbf{isi\_en.19.X.12.S4}]
\textit{Green silex is itself stubbornly resistant to fire, but there is nowhere where it is abundant, and it is found only as a stone and not as a rocky outcrop.}
\end{itemize}
\item [Livro: 4 - Medicine (15.67\%)]\hfill
\begin{itemize}
\item[\textbf{isi\_en.4.IX.8.S1}]
\textit{The remedy hiera is so called as if it were 'holy,' (cf. ¬?póç, "holy").}
\item[\textbf{isi\_en.4.VI.16.S0}]
\textit{A carbuncle (carbunculus) is so called because at first it glows red, like fire, and then turns black, like an extinguished coal (carbo).}
\item[\textbf{isi\_en.4.VII.25.S0}]
\textit{Paralysis (paralesis) is so called from damage to the body caused by excessive chilling, occurring either in the entire body, or in one part.}
\item[\textbf{isi\_en.4.VIII.3.S1}]
\textit{Latin speakers call erysipelas (erisipela) 'sacred fire' - speaking in antiphrasis, as it should be cursed - inasmuch as the skin grows flame-red on its surface.}
\item[\textbf{isi\_en.4.X.3.S0}]
\textit{A dinamidia describes the power of herbs, that is, their force and capability.}
\end{itemize}
\end{description}
