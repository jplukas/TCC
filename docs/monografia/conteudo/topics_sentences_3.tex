\subsubsection{Tópico 3: \textit{garment, hair, wear, cloak, woven, cloth, linen, rings, tunic, silk}}
\label{tema:sentences_3}
\begin{description}
\footnotesize
\item [Livro: 19 - Ships, buildings, and clothing (33.75\%)]\hfill
\begin{itemize}
\item[\textbf{isi\_en.19.I.27.S1}]
\textit{It is also known as the litoraria, and as the caudica, made from a single hollowed piece of wood (cf.}
\item[\textbf{isi\_en.19.V.3.S1}]
\textit{It is also called the 'seine' (verriculum) because verrere means "drag."}
\item[\textbf{isi\_en.19.VII.1.S4}]
\textit{tussus) on it, that is, stretched out.}
\item[\textbf{isi\_en.19.XVII.6.S0}]
\textit{Now, 'silk' (sericum) is one thing, and 'Syrian' (Syricum) is another, for silk is a fiber that the Chinese (Seres; East Asians generally) export, while Syrian is a pigment that the Syrian Phoenicians gather at the shores of the Red Sea.}
\item[\textbf{isi\_en.19.XVIII.3.S0}]
\textit{A string (linea) is named from its material, for it is made from flax (linum).}
\end{itemize}
\item [Livro: 11 - The human being and portents (11.24\%)]\hfill
\begin{itemize}
\item[\textbf{isi\_en.11.I.26.S0}]
\textit{The crown (vertex) is the place where the hairs of the head concentrate, and where the hair growth spirals (vertere, "turn") - whence it is named.}
\end{itemize}
\item [Livro: 20 - Provisions and various implements (5.50\%)]\hfill
\begin{itemize}
\item[\textbf{isi\_en.20.III.14.S1}]
\textit{It is called passum from 'suffering' (pati, ppl.}
\item[\textbf{isi\_en.20.IX.5.S0}]
\textit{Bag (saccus) is so called from the word 'blanket' (sagum), because it is made by sewing one up, as if the word were sagus.}
\item[\textbf{isi\_en.20.V.5.S1}]
\textit{An ampulla (ampulla) is as if the word were a 'large bubble' (ampla bulla), for it has the spherical shape of bubbles that are made from foaming water and thus are inflated by the wind.}
\item[\textbf{isi\_en.20.VII.1.S1}]
\textit{A scortia is a vessel for oil, so called because it is made of leather (corium).}
\item[\textbf{isi\_en.20.XII.4.S4}]
\textit{Unless they were chaste, matrons could not use these; nor, likewise, could they wear fillets.}
\end{itemize}
\item [Livro: 18 - War and games (5.31\%)]\hfill
\begin{itemize}
\item[\textbf{isi\_en.18.LXIX.1.S0}]
\textit{A ball (pila) is properly so called because it is stuffed with hair (pilus).}
\item[\textbf{isi\_en.18.XII.4.S0}]
\textit{A peltum (i.e. pelta) is a very short buckler shaped like a half moon.}
\item[\textbf{isi\_en.18.XIII.2.S0}]
\textit{2. 'Scale-armor' (squama) is an iron cuirass made from iron or bronze plates linked together in the manner of fish scales (squama), and named for their glittering likeness to fish scales.}
\item[\textbf{isi\_en.18.XIV.1.S1}]
\textit{The cassis was named by the Etruscans, and I think they named that helmet cassis from the word 'head' (caput).}
\end{itemize}
\item [Livro: 5 - Laws and times (2.41\%)]\hfill
\begin{itemize}
\item[\textbf{isi\_en.5.XXVII.7.S0}]
\textit{. Foot-shackles (compes) are named because they 'restrain the feet' (continere + pes).}
\end{itemize}
\end{description}
