\subsubsection{Tópico 4: \textit{person, mind, term, ones, ppl, dead, does, sense, body, ancients}}
\label{tema:topics_4}
\begin{description}
\footnotesize
\item [Livro: 10 - Vocabulary (83.64\%)]\hfill
\begin{itemize}
\item[\textbf{isi\_en.10.A.11.Sall}]
\textit{11. Indecisive (anceps), wavering this way and that and doubting whether to choose this or that, and distressed (anxius) about which way to lean. Abominable (atrox), because one has loathsome (taeter) conduct. Abstemious (abstemius), from temetum, that is, 'wine,' as if abstaining (abstinere) from wine. [Neighboring (affinis) . . .] Weaned (ablactatus), because one is 'withdrawn from milk' (a lacte ablatus). }
\item[\textbf{isi\_en.10.B.31.Sall}]
\textit{31. Baburrus, "stupid, inept." Biothanatus (i.e. a martyr who dies a violent death), because he is 'twice dead,' for death is 9?vatoç in Greek.  }
\item[\textbf{isi\_en.10.C.40.Sall}]
\textit{40. Constant (constans) is so called because one 'stands firm' (stare, present participle stans) in every situation, and cannot deviate in any direction. Trusting (confidens), one full of faith (fiducia) in all matters. Whence Caecilius (fr. 246): If you summon Confidence, confide (confidere) everything to her. }
\end{itemize}
\item [Livro: 5 - Laws and times (13.59\%)]\hfill
\begin{itemize}
\item[\textbf{isi\_en.5.XXVI.1.Sall}]
\textit{1. Crime (crimen) has its name from lacking (carere) - like theft, deceit, and other actions that do not kill, but cause disgrace. }
\item[\textbf{isi\_en.5.XXVII.1.Sall}]
\textit{1. Harm (malum) is defined in two ways: one definition being what a person does; the other, what he suffers. What he does is wrongdoing (peccatum), what he suffers is punishment. And harm is at its full extent when it is both past and also impending, so that it includes both grief and dread. }
\end{itemize}
\item [Livro: 11 - The human being and portents (10.47\%)]\hfill
\begin{itemize}
\item[\textbf{isi\_en.11.I.1.Sall}]
\textit{1. Nature (natura) is so called because itcauses something to be born (nasci, ppl. natus), for it has the power of engendering and creating. Some people say that this is God, by whom all things have been created and exist. }
\item[\textbf{isi\_en.11.II.31.Sall}]
\textit{31. Death (mors) is so called, because it is bitter (amarus), or by derivation from Mars, who is the author of death; [or else, death is derived from the bite (morsus) of the first human, because when he bit the fruit of the forbidden tree, he incurred death]. }
\item[\textbf{isi\_en.11.III.27.Sall}]
\textit{27. They claim also that in the same India is a race of women who conceive when they are five years old and do not live beyond eight. }
\end{itemize}
\item [Livro: 6 - Books and ecclesiastical offices (9.90\%)]\hfill
\begin{itemize}
\item[\textbf{isi\_en.6.XIX.35.Sall}]
\textit{35. A holocaust (holocaustum) is a sacrifice in which all that is offered is consumed by fire, for when the ancients would perform their greatest sacrifices, they would consume the whole sacrificial victim in the flame of the rites, and those were holocausts, for o2oç in Greek means "whole," mauotç means "burning," and holocaust, "wholly burnt." }
\end{itemize}
\item [Livro: 8 - The Church and sects (4.37\%)]\hfill
\begin{itemize}
\item[\textbf{isi\_en.8.II.7.Sall}]
\textit{7. It is greater than the other two, because he who loves also believes and hopes. But he who does not love, although he may do many good things, labors in vain. Moreover every carnal love (dilectio carnalis) is customarily called not love (dilectio) but 'desire' (amor). We usually use the term dilectio only with regard to better things. }
\item[\textbf{isi\_en.8.III.6.Sall}]
\textit{6. Superstition (superstitio) is so called because it is a superfluous or superimposed (superinstituere) observance. Others say it is from the aged, because those who have lived (superstites) for many years are senile with age and go astray in some superstition through not being aware of which ancient practices they are observing or which they are adding in through ignorance of the old ones. }
\item[\textbf{isi\_en.8.IV.11.Sall}]
\textit{11. The Hemerobaptistae [who wash their bodies and home and domestic utensils daily,] [so called because they wash their clothes and body daily (cf. ¡µspa, "day," and ßapt(c)S?tv, "to wash")]. }
\end{itemize}
\end{description}
