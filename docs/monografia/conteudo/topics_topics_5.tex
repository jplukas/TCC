\subsubsection{Tópico 5: \textit{tree, grows, fruit, root, leaves, plant, flower, herb, olive, leaves like}}
\label{tema:topics_5}
\begin{description}
\footnotesize
\item [Livro: 17 - Rural matters (79.26\%)]\hfill
\begin{itemize}
\item[\textbf{isi\_en.17.III.15.Sall}]
\textit{15. We speak improperly of the 'ear' (spica) of ripe fruit, for properly the ear exists when the beards, still thin like spear-tips (spiculum), project through the husk of the stalk, that is the swelling tip. }
\item[\textbf{isi\_en.17.IV.6.Sall}]
\textit{6. 'French bean' (faselum; cf. ???o?2oç) and chickpea (cicer; cf. mptóç) are Greek names. But faselum . . . }
\item[\textbf{isi\_en.17.IX.105.Sall}]
\textit{105.A fern (filix) is so called from the singleness of its leaf (folium, cf. filum, "a single strand"), for from one stalk a cubit high grows one divided leaf, with an intricate structure like a feather's. Oats (avena) . . . Darnel (lolium) .  }
\end{itemize}
\item [Livro: 4 - Medicine (9.38\%)]\hfill
\begin{itemize}
\item[\textbf{isi\_en.4.IX.9.Sall}]
\textit{9. Catapotia, because a little is drunk (potare) or swallowed down. Diamoron got its name from the juice of the mulberry (morum), from which it is made; likewise diacodion, because it is made from the poppy-head (codia; cf. mÛ6?ta, "poppyhead"), that is, from the poppy; and similarly diaspermaton, because it is made from seeds (cf. opspµa, "seed"). }
\item[\textbf{isi\_en.4.X.3.Sall}]
\textit{3. A dinamidia describes the power of herbs, that is, their force and capability. In herbal medicine, potency itself is called 6ávaµtç, whence also the books where herbal remedies are inscribed are called dinamidia.A 'botanical treatise' (butanicum, i.e. botanicum, cf. ßot?v?, "herb") about plants is so called because plants are described in it. }
\item[\textbf{isi\_en.4.XII.2.Sall}]
\textit{2. It is called incense (thymiama) in the Greek language, because it is scented, for a flower that bears a scent is called thyme (thymum). With regard to this, Vergil (Geo. 4.169): And (the honey) is redolent with thyme. }
\end{itemize}
\item [Livro: 16 - Stones and metals (1.59\%)]\hfill
\begin{itemize}
\item[\textbf{isi\_en.16.II.8.Sall}]
\textit{8. The Greek term aphronitrum is 'foam of natron,' spuma nitri in Latin. Concerning this a certain poet says (Martial, Epigrams 14.58): Are you a bumpkin? You don't know what my Greek name is. I am called 'foam of natron.' Are you Greek? It is aphronitrum. It is gathered in Asia where it distills in caves; from there it is dried in the sun. It is thought to be best if it has as little weight and is as easily crumbled as possible, and is almost purple in color. }
\item[\textbf{isi\_en.16.IV.14.Sall}]
\textit{14. Memphitis is named from a place in Egypt (i.e. Memphis); it has the nature of a gem. When ground and mixed with vinegar and smeared on those parts of the body that have to be burned or cut it makes them so numb that they do not feel pain. }
\item[\textbf{isi\_en.16.VII.14.Sall}]
\textit{14. Myrrhites is so named because it has the color of myrrh. When it is compressed until it becomes warm it exudes the sweet smell of nard. Aromatitis is found in Arabia and Egypt; it has the color and scent of myrrh, whence it takes its name (cf. aroma, "spice"). }
\end{itemize}
\item [Livro: 12 - Animals (1.37\%)]\hfill
\begin{itemize}
\item[\textbf{isi\_en.12.VI.37.Sall}]
\textit{37. The squatus is named because it has 'sharp scales' (squamis acutus). For this reason wood is polished with its skin. }
\item[\textbf{isi\_en.12.VII.23.Sall}]
\textit{23. The cinnamolgus is also a bird of Arabia, called thus because in tall trees it constructs nests out of cinnamon (cinnamum) shrubs, and since humans are unable to climb up there due to the height and fragility of the branches, they go after the nests using lead-weighted missiles. Thus they dislodge these cinnamon nests and sell them at very high prices, for merchants value cinnamon more than other spices. }
\item[\textbf{isi\_en.12.VIII.8.Sall}]
\textit{8. Butterflies (papilio) are small flying creatures that are very abundant when mallows bloom, and they cause small worms to be generated from their own dung. }
\end{itemize}
\item [Livro: 20 - Provisions and various implements (1.22\%)]\hfill
\begin{itemize}
\item[\textbf{isi\_en.20.II.36.Sall}]
\textit{36. Honey (mel) is froma Greek term (i.e. µs2t), which is shown to have its name from bees, for a bee in Greek is called µs2tooa. Formerly honey was from dew and was found in the leaves of reeds. Hence Vergil (Geo. 4.1): Forthwith, the celestial gifts of honey from the air. Indeed in India and Arabia honey is still found attached to branches, clinging like grains of salt. Nevertheless, all honey is sweet; Sardinian honey is bitter because of wormwood, with whose abundance the bees of that region are fed. The honeycomb (favus) is so called because it is eaten rather than drunk, for the Greeks say ??ay?±v for "eat." }
\item[\textbf{isi\_en.20.VII.3.Sall}]
\textit{3. A pyx (pyxis) is a little container for ointment made of boxwood, for what we call 'boxwood' the Greeks call pá(oç. }
\end{itemize}
\end{description}
