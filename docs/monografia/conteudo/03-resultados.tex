%!TeX root=../monografia.tex
%("dica" para o editor de texto: este arquivo é parte de um documento maior)
% para saber mais: https://tex.stackexchange.com/q/78101

\chapter{Resultados}
\label{cap:resultados}
% Cap 3 (5 páginas): Resultados. Aqui você pode começar a colocar figuras com os resultados dos experimentos do Cap 3, explicando basicamente o que você fez de variar os diferentes parâmetros e como isso influenciou no resultado, essas coisas. Depois podemos colocar também resultados obtidos com as buscas e potenciais cruzamentos interessantes do ponto de vista dos historiadores.
A seguir relataremos os resultados dos experimentos realizados.

\begin{figure}[h]
    \includegraphics[width=\textwidth]{embspace}
    \caption{Visualização 2D do espaço de embeddings.}
\end{figure}

Ao executar o algoritmo do BERTopic com diferentes parâmetros, obtivemos diferentes resultados. Em geral, o número de pontos classificados como ruído, ou outliers, foi alto. Porém, ao inspecionar as configurações que minimizaram essa quantidade, descobrimos que algumas geravam agrupamentos muito concentrados em poucos clusters, misturando temas diferentes. A solução adotada foi escolher as configurações que nos dessem clusters mais homogêneos, minimizando o tamanho máximo de um cluster. Essa abordagem também acabou gerando um número reduzido de outliers, como pode-se ver nas figuras \ref{fig:outliers} e \ref{fig:maxclus}.

A segmentação do texto em seções gerou tópicos mais coerentes na maioria dos casos. Porém, algumas seções de maior comprimento contribuíram para clusters impuros, com temas mistos. Em contrapartida, as configurações que segmentaram o texto em sentenças criaram mais clusters, com maior concentração das palavras em clusters dominantes.

\begin{figure}[h]
    \includegraphics[width=\textwidth]{comp-outliers}
    \caption{Contagem de documentos por tópico - minimização de outliers.}
    \label{fig:outliers}
\end{figure}

\begin{figure}[h]
    \includegraphics[width=\textwidth]{comp-maxclus}
    \caption{Contagem de documentos por tópico - minimização de cluster máximo.}
    \label{fig:maxclus}
\end{figure}

Na figura \ref{fig:livro-topico}, podemos observar a presença tanto de temas concentrados, como o primeiro, que contém temas de gramática, dialética e retórica, quanto temas mais transversais, distribuídos em diferentes livros.
\begin{figure}[h]
    \includegraphics[width=\textwidth]{dist_livros}
    \caption{Composição de livros por tópico.}
    \label{fig:livro-topico}
\end{figure}