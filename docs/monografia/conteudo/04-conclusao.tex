%!TeX root=../monografia.tex
%("dica" para o editor de texto: este arquivo é parte de um documento maior)
% para saber mais: https://tex.stackexchange.com/q/78101

\chapter{Conclusão}
\label{cap:conclusao}

Neste trabalho exploramos a intersecção entre Computação e História, investigando o uso de ferramentas de Processamento de Linguagem Natual como Grandes Modelos de Linguagem (LLMs) e técnicas de modelagem de tópicos para a análise de textos históricos, especificamente as \emph{Etimologias} de Isidoro de Sevilha e textos relacionados. O objetivo principal foi o de mapear temas transversais e estruturas semânticas latentes que são custosos e difíceis de serem realizados por meio de análise manual.

Os experimentos realizados demonstraram que a utilização do \emph{pipeline} proposto, baseado em \emph{embeddings} gerados por LLMs, é eficaz para organizar e sumarizar grandes \emph{corpora} de textos históricos. A análise dos resultados da modelagem de tópicos revelou um \emph{trade-off} importante entre a minimização de outliers e a pureza dos clusters. Observamos que configurações que forçam uma redução drástica de ruído tendem a criar agrupamentos heterogêneos, mesclando temas distintos. A abordagem mais eficaz consistiu em minimizar o tamanho máximo dos clusters, o que resultou em tópicos mais coerentes e uma representação mais fiel da estrutura das obras estudadas. Além disso, a comparação entre diferentes granularidades mostrou que a segmentação por sentenças permitiu uma concentração temática superior à segmentação por seções, isolando conceitos específicos com maior precisão.

Do ponto de vista historiográfico e literário, a visualização do espaço de \emph{embeddings} confirmou a existência de temas altamente concentrados, como Gramática, Retórica e Dialética, alinhados à estrutura educacional do \emph{trivium} e \emph{quadrivium}. Mais importante ainda, foi possível identificar temas transversais que permeiam diferentes livros da obra, validando a capacidade do modelo de capturar conexões semânticas para além da organização explícita proposta por Isidoro. As nuvens de palavras geradas (e.g., termos agrícolas, conceitos teológicos e materiais) evidenciam a capacidade do modelo de recuperar o vocabulário distintivo de cada tópico.

Em suma, este estudo conclui que, embora o uso de LLMs e modelagem de tópicos não substituam a leitura crítica do especialista, ele oferece uma ferramenta poderosa de "leitura distante" e prospecção. As técnicas empregadas conseguiram não apenas replicar a categorização temática humana, mas também sugerir novas conexões semânticas. Trabalhos futuros podem se beneficiar do refinamento dos parâmetros de clustering para reduzir ainda mais a taxa de outliers sem sacrificar a coerência, bem como aprofundar a análise comparativa de similaridade direta entre as sentenças de Isidoro e os textos de Catão e Varrão para identificar citações não atribuídas com maior precisão.
