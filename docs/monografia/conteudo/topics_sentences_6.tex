\subsubsection{Tópico 6: \textit{disease, bile, intestines, accompanied, blood, lungs, pain, illness, fever, phlegm}}
\label{tema:sentences_6}
\begin{description}
\footnotesize
\item [Livro: 4 - Medicine (49.31\%)]\hfill
\begin{itemize}
\item[\textbf{isi\_en.4.IX.11.S3}]
\textit{Enema (enema) in Greek is called 'loosening' (relaxatio) in Latin.}
\item[\textbf{isi\_en.4.V.1.S0}]
\textit{Health is integrity of the body and a balance of its nature with respect to its heat and moisture, which is its blood - hence health (sanitas) is so called, as if it were the condition of the blood (sanguis).}
\item[\textbf{isi\_en.4.VI.1.S0}]
\textit{An ò(?±a is an acute illness that either passes quickly or kills rather quickly, such as pleurisis, or phrenesis, for in Greek ò(áç means acute and swift.}
\item[\textbf{isi\_en.4.VII.1.S0}]
\textit{Chronic disease (chronia) is an extended illness that lasts for a long time, like gout or consumption, for ypóvoç in Greek means "time."}
\item[\textbf{isi\_en.4.VIII.2.S0}]
\textit{2. Parotids (parotida) are areas of hardness or accretions that emerge in the vicinity of the ears, caused by fever or something else, whence they are called papYt(c)6?ç, for Yta is the Greek word for ears.}
\end{itemize}
\item [Livro: 11 - The human being and portents (17.60\%)]\hfill
\begin{itemize}
\item[\textbf{isi\_en.11.I.57.S0}]
\textit{In the Gallic language toles (cf. classical Latin toles, "goiter") - what in the diminutive are commonly called tonsils (tusilla, i.e. tonsilla) - is the name for the part in the throat that often swells up (turgescere).}
\item[\textbf{isi\_en.11.II.30.S3}]
\textit{Indeed, there are two things whereby the forces of the body are diminished, old age and disease.}
\end{itemize}
\item [Livro: 20 - Provisions and various implements (8.93\%)]\hfill
\begin{itemize}
\item[\textbf{isi\_en.20.I.3.S5}]
\textit{Eating and drinking, as (cf.}
\item[\textbf{isi\_en.20.II.2.S1}]
\textit{Youths take nourishment in order to grow, the elderly to endure, for the flesh cannot subsist unless it is strengthened with nourishment.}
\item[\textbf{isi\_en.20.IV.13.S0}]
\textit{Spoons (coclear) are so called because they were first used for snails (coclea).}
\item[\textbf{isi\_en.20.VIII.1.S0}]
\textit{Any vessel intended for cooking (coquere) is called coculum.}
\end{itemize}
\item [Livro: 6 - Books and ecclesiastical offices (3.04\%)]\hfill
\begin{itemize}
\item[\textbf{isi\_en.6.XIX.65.S0}]
\textit{Fasting (ieiunium) is parsimony of sustenance and abstinence from food, and its name is given to it from a certain portion of the intestines, always thin and empty, which is commonly called the jejunum (ieiunum).}
\end{itemize}
\item [Livro: 10 - Vocabulary (1.31\%)]\hfill
\begin{itemize}
\item[\textbf{isi\_en.10.C.61.S0}]
\textit{Epileptic (caducus, "falling sickness"), so called from falling down (cadere).}
\item[\textbf{isi\_en.10.D.70.S0}]
\textit{70. 'Thickly smeared' (delibutus), anointed with oil as is the custom for athletes or youths in the wrestling arena.}
\item[\textbf{isi\_en.10.F.99.S1}]
\textit{and] blood arouses beauty.}
\item[\textbf{isi\_en.10.L.161.S0}]
\textit{Panic-stricken (lymphaticus), because one fears water (cf. lympha, "water"), one whom the Greeks call ú6po??óßoç ("hydrophobic").}
\item[\textbf{isi\_en.10.M.176.S1}]
\textit{Evil (malus) is named after black bile; the Greeks call black µs2aç.}
\end{itemize}
\end{description}
