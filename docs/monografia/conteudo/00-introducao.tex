%!TeX root=../monografia.tex
%("dica" para o editor de texto: este arquivo é parte de um documento maior)
% para saber mais: https://tex.stackexchange.com/q/78101

%% ------------------------------------------------------------------------- %%

% "\chapter" cria um capítulo com número e o coloca no sumário; "\chapter*"
% cria um capítulo sem número e não o coloca no sumário. A introdução não
% deve ser numerada, mas deve aparecer no sumário. Por conta disso, este
% modelo define o comando "\chapter**".
\chapter{Introdução}
\label{cap:introducao}
% Cap 0 (2 páginas): Introdução, objetivos e contribuição. A gente pode aproveitar parte do texto que está na proposta do TCC para isso.

Tradicionalmente, pesquisadores das áreas de humanidades como história, filosofia, entre outras, dependem da análise de grandes quantidades de fontes para a realização do seu trabalho. Esta análise tipicamente é realizada de forma manual, e com grande custo, tanto em termos de tempo quanto de recursos humanos, custo este que pode se tornar um fator limitador da produção científica desses pesquisadores.

A área de processamento de linguagem natural (NLP) possui métodos bem estabelecidos para análise semântica de textos, como por exemplo, modelagem de tópicos~\parencite{deerwester1990indexing,blei2003latent,jelodar2019latent}, que visa agrupar textos que tratam de assuntos semelhantes. Com o surgimento dos grandes modelos de linguagem (LLM) baseados em transformers~\parencite{vaswani2017attention,devlin2019bert}, é possível observar um grande salto em termos de qualidade das ferramentas e métodos para análise de texto, o que pode ser atribuído à maior capacidade de mapeamento de conceitos em um espaço latente de atributos, que, por sua vez, possibilita um melhor agrupamento de textos em termos de semântica. Em todo caso, o uso destas técnicas requer certa familiaridade com o seu funcionamento e com os textos analisados, de modo que sejam feitos os ajustes necessários para a obtenção de bons resultados.

\section{Objetivos}

Neste trabalho pretendemos, por meio do uso de técnicas computacionais de processamento de texto, estudar os tópicos existentes na obra conhecida como \emph{Etimologias}, de Isidoro de Sevilha (c.560-636). Esta obra, que é considerada como a primeira grande enciclopédia da Idade Média, é formada por 20 livros que tratam das origens das palavras agrupadas em diversos grandes temas. O autor, que viveu em uma época de grandes mudanças culturais, buscava registrar o conhecimento de escritores latinos da Antiguidade Clássica, como Varrão, Catão e Plínio o Velho, entre outros. A obra, que foi escrita em latim medieval, foi copiada exaustivamente ao longo de cerca de 700 anos, sendo utilizada como uma espécie de livro-texto básico nas instituições de ensino da época. Por conveniência, para este trabalho será utilizada uma tradução recente para a língua inglesa~\parencite{barney2006etymologies}.

\section{Contribuições}

A principal contribuição pretendida com este trabalho é conseguir mapear temas transversais discutidos na obra estudada, para além dos temas gerais propostos pelo autor. Adicionalmente, propomos um experimento de análise de similaridade semântica entre a obra estudada e possíveis fontes da antiguidade, para identificar potenciais citações sem atribuição.

\section{Organização}

O texto está organizado da seguinte forma: no Capítulo~\ref{cap:background} tratamos do background histórico dos autores estudados, bem como das ferramentas utilizadas; no Capítulo~\ref{cap:experimentos} apresentamos os dados analisados, o \emph{pipeline} de processamento criado para os dados, e a configuração dos experimentos realizados; no Capítulo~\ref{cap:resultados} apresentamos e discutimos os resultados obtidos para cada experimento realizado. O Capítulo~\ref{cap:conclusao} conclui o texto.
