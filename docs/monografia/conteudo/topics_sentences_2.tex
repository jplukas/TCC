\subsubsection{Tópico 2: \textit{bird, birds, animal, venom, poison, owl, offspring, animals, fly, lions}}
\label{tema:sentences_2}
\begin{description}
\footnotesize
\item [Livro: 12 - Animals (69.86\%)]\hfill
\begin{itemize}
\item[\textbf{isi\_en.12.I.12.S0}]
\textit{Although the Greeks name the lamb (agnus) from ?yvóç ("holy") as if it were sacred, Latin speakers think that it has this name because it recognizes (agnoscere) its mother before other animals, to the extent that even if it has strayed within a large herd, it immediately recognizes the voice of its parent by its bleat.}
\item[\textbf{isi\_en.12.II.1.S1}]
\textit{They are called beasts (bestia) from the force (vis) with which they attack.}
\item[\textbf{isi\_en.12.III.9.S1}]
\textit{It has great shrewdness, for it provides for the future and prepares during the summer what it consumes in the winter; during the harvest it selects the wheat and does not touch the barley.}
\item[\textbf{isi\_en.12.IV.4.S1}]
\textit{The Greeks call it 6p?mYv, whence the term is borrowed into Latin so that we say draco.}
\item[\textbf{isi\_en.12.V.7.S0}]
\textit{. Slugs (limax) are mud vermin, so named because they are generated either in mud (limus) or from mud; hence they are always regarded as filthy and unclean.}
\end{itemize}
\item [Livro: 11 - The human being and portents (7.87\%)]\hfill
\begin{itemize}
\item[\textbf{isi\_en.11.I.39.S2}]
\textit{They meet together to renew the power of the gaze with their frequent motion.}
\item[\textbf{isi\_en.11.II.25.S0}]
\textit{The 'elder' (senior) is still fairly vigorous.}
\item[\textbf{isi\_en.11.III.26.S0}]
\textit{In India there are said to be a race called Mampóßtot, who are twelve feet tall.}
\end{itemize}
\item [Livro: 20 - Provisions and various implements (5.50\%)]\hfill
\begin{itemize}
\item[\textbf{isi\_en.20.II.23.S3}]
\textit{Rancid (rancidus) is named after its defect, because it makes meat harsh (raucus).}
\item[\textbf{isi\_en.20.III.2.S2}]
\textit{The ancients called wine venom (venenum), but after poison from a lethal sap was discovered they called the one thing wine, the other venom.}
\item[\textbf{isi\_en.20.XII.4.S1}]
\textit{): Mothers in soft carriages (pilentum).}
\item[\textbf{isi\_en.20.XIV.8.S0}]
\textit{A hoe (sarculus)- of these there are the single-bladed and the two-pronged.}
\item[\textbf{isi\_en.20.XV.4.S0}]
\textit{The 'wolf' (lupus) or 'little dog' (canicula) is an iron grapple that takes such names because if anything falls in a well it snatches them and draws them out.}
\end{itemize}
\item [Livro: 8 - The Church and sects (3.69\%)]\hfill
\begin{itemize}
\item[\textbf{isi\_en.8.IX.10.S0}]
\textit{Hence also Lucan (Civil War 6.457): The mind, polluted by no poison of swallowed venom, yet perishes under a spell.}
\item[\textbf{isi\_en.8.V.69.S4}]
\textit{Some walk barefoot, others do not eat with other people.}
\item[\textbf{isi\_en.8.XI.63.S2}]
\textit{They furnish her with tame lions below to show that no kind of beast is so wild that it cannot be subjugated and ruled by her.}
\end{itemize}
\item [Livro: 18 - War and games (2.42\%)]\hfill
\begin{itemize}
\item[\textbf{isi\_en.18.III.3.S0}]
\textit{The standard of dragons originated in the killing of the serpent Python by Apollo.}
\item[\textbf{isi\_en.18.VII.4.S1}]
\textit{Indeed, "Theyintercept (excipere) boars, lie in wait for lions, and penetrate bears, if only one's hand is steady" (cf.}
\item[\textbf{isi\_en.18.X.2.S1}]
\textit{It is wound up with a thong of sinew and hurls either spears or stones with great force.}
\item[\textbf{isi\_en.18.XII.5.S2}]
\textit{): A cetra covers their left arms.}
\end{itemize}
\end{description}
