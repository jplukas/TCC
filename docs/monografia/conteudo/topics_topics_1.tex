\subsubsection{Tópico 1: \textit{words, letter, letters, aen, vergil aen, speech, vergil, cicero, nouns, syllable}}
\label{tema:topics_1}
\begin{description}
\footnotesize
\item [Livro: 2 - Rhetoric and dialectic (85.78\%)]\hfill
\begin{itemize}
\item[\textbf{isi\_en.2.I.1.Sall}]
\textit{1. Rhetoric is the art of speaking well in civil cases, [and eloquence (eloquentia) is fluency (copia)] for the purpose of persuading people toward the just and good. Rhetoric is named from the Greek term p?top(c)S?tv, that is, fluency of speech, for p?otç in Greek means "speech," p?tYp means "orator." }
\item[\textbf{isi\_en.2.II.2.Sall}]
\textit{2. For while one has a treatise on rhetoric in hand, the sequence of its content as it were clings to the memory, but when it is set aside all recollection of it soon slips away. Accomplished knowledge of this discipline makes one an orator. }
\item[\textbf{isi\_en.2.III.1.Sall}]
\textit{1. An orator therefore is a good man, skilled in speaking. A man's goodness is based on his nature, his behavior, his training in the arts. One skilled in speaking is grounded in artful eloquence, which consists of five parts: invention, arrangement, style, memory, pronunciation (inventio, dispositio, elocutio, memoria, pronuntiatio), and of the goal of this office, which is to persuade of something. }
\end{itemize}
\item [Livro: 1 - Grammar (82.79\%)]\hfill
\begin{itemize}
\item[\textbf{isi\_en.1.II.1.Sall}]
\textit{1. There are seven disciplines of the liberal arts. The first is grammar, that is, skill in speaking. The second is rhetoric, which, on account of the brilliance and fluency of its eloquence, is considered most necessary in public proceedings. The third is dialectic, otherwise known as logic, which separates the true from the false by very subtle argumentation. }
\item[\textbf{isi\_en.1.III.1.Sall}]
\textit{1. The common letters of the alphabet are the primary elements of the art of grammar, and are used by scribes and accountants. The teaching of these letters is, as it were, the infancy of grammar, whence Varro also calls this discipline 'literacy' (litteratio). Indeed, letters are tokens of things, the signs of words, and they have so much force that the utterances of those who are absent speak to us without a voice, [for they present words through the eyes, not through the ears]. }
\item[\textbf{isi\_en.1.IV.6.Sall}]
\textit{6. Now they are vowels, and now semivowels, and now medials (i.e. glides). They are vowels because they make syllables when they are positioned alone or when they are joined to consonants. They are considered consonants in that they sometimes have a vowel set down after them in the same syllable, as Ianus, vates, and they are considered as consonants. }
\end{itemize}
\item [Livro: 6 - Books and ecclesiastical offices (7.43\%)]\hfill
\begin{itemize}
\item[\textbf{isi\_en.6.IX.2.Sall}]
\textit{2. Hence it was said among scribes, "You shall not strike wax with iron." Afterwards it was established that they would write on wax tablets with bones, as Atta indicates in his Satura, saying (12): Let us turn the plowshare and plow in the wax with a point of bone. The Greek term graphium is scriptorium in Latin, for ypa??? is "writing." }
\item[\textbf{isi\_en.6.VIII.7.Sall}]
\textit{7. A panegyric (panegyricum) is an extravagant and immoderate form of discourse in praise of kings; in its composition people fawn on them with many lies. This wickedness had its origin among the Greeks, whose practised glibness in speaking has with its ease and incredible fluency stirred up many clouds of lies. }
\item[\textbf{isi\_en.6.XIV.2.Sall}]
\textit{2. The scribe (scriba) got his name from writing (scribere), expressing his function by the character of his title. }
\end{itemize}
\item [Livro: 5 - Laws and times (4.35\%)]\hfill
\begin{itemize}
\item[\textbf{isi\_en.5.XXV.32.Sall}]
\textit{32. Cessio is a concession (concessio) of one's own property, such as this: "I cede by right of affinity," for we say 'cede' (cedere) as if it were 'concede' (concedere), that is, those things that are our own; for we 'restore' the property of another, we do not 'cede' it. In fact, technically speaking, someone is said to cede when he gives in to another in spite of the truth, as Cicero (Defense of Ligarius 7.22): "He ceded," he says, "to the authority of a very distinguished man, or rather, he obeyed." }
\item[\textbf{isi\_en.5.XXVI.9.Sall}]
\textit{9. Deceit (falsitas) is so called from saying (fari, ppl. fatus) something other than the truth. }
\item[\textbf{isi\_en.5.XXVII.27.Sall}]
\textit{27. Report does not possess a trustworthy name, because it is especially untruthful, either adding many things to the truth, or distorting the truth. It lasts just as long as it is not put to the test, but whenever you put it to the test, it ceases to be, and after that is called fact, not report. }
\end{itemize}
\item [Livro: 8 - The Church and sects (1.98\%)]\hfill
\begin{itemize}
\item[\textbf{isi\_en.8.IX.29.Sall}]
\textit{29. The salisatores are so called because whenever any part of their limbs leaps (salire), they proclaim that this means something fortunate or something unfortunate for them thereafter. }
\item[\textbf{isi\_en.8.VII.5.Sall}]
\textit{5. Tragedians (tragoedus) are so called, because at first the prize for singers was a goat, which the Greeks call tp?yoç. Hence also Horace (Art of Poetry 220): Who with a tragic song vied for a paltry goat. Now the tragedians following thereafter attained great honor, excelling in the plots of their stories, composed in the image of truth. }
\end{itemize}
\end{description}
