\subsubsection{Tópico 2: \textit{city, river, sea, founded, east, region, west, north, south, island}}
\label{tema:topics_2}
\begin{description}
\footnotesize
\item [Livro: 14 - The earth and its parts (87.17\%)]\hfill
\begin{itemize}
\item[\textbf{isi\_en.14.II.3.Sall}]
\textit{3. Whence it is clear that two of them, Europe and Africa, occupy half of the globe, Asia the other half by itself. But the former pair are divided into two regions, because from the Ocean the Mediterranean enters in between them and separates them. Wherefore, if you divide the globe into two parts, the east and the west, Asia will be in one, Europe and Africa in the other. }
\item[\textbf{isi\_en.14.III.46.Sall}]
\textit{46. Lycia is so called because in the east it borders on Cilicia, for Cilicia borders on it in the east and it has the sea to the west and the south; in the north lies Caria. There lies Mount Chimera, which exhales fire in nightly surges, like Etna in Sicily and Vesuvius in Campania. }
\item[\textbf{isi\_en.14.IV.30.Sall}]
\textit{30. Furthermore there are two Spains: Inner Spain, whose area extends in the north from the Pyrenees to Cartagena; and Outer Spain, which in the south extends from Celtiberia to the straits of Cadiz. Inner (citerior) and Outer (ulterior) are so called as if it were citra (on this side) and ultra (beyond); but citra is formed as if the term were 'around the earth' (circa terras), and ultra either because it is the last (ultimus), or because after it there is not 'any' (ulla), that is, any other, land. }
\end{itemize}
\item [Livro: 9 - Languages, nations, reigns, the military, citizens, family relationships (41.58\%)]\hfill
\begin{itemize}
\item[\textbf{isi\_en.9.II.2.Sall}]
\textit{2. Now, of the nations into which the earth is divided, fifteen are from Japheth, thirty-one from Ham, and twenty-seven from Shem, which adds up to seventythree - or rather, as a proper accounting shows, seventytwo. And there are an equal number of languages, which arose across the lands and, as they increased, filled the provinces and islands. }
\item[\textbf{isi\_en.9.III.2.Sall}]
\textit{2. Every nation has had its own reign in its own times - like the Assyrians, the Medes, the Persians, the Egyptians, the Greeks - and fate has so rolled over their allotments of time that each successive one would dissolve the former. Among all the reigns on earth, however, two reigns are held to be glorious above the rest - first the Assyrians, then the Romans - as they are constituted differently from one another in location as much as time. }
\item[\textbf{isi\_en.9.IV.28.Sall}]
\textit{28. Burghers (burgarius) are so called from 'fortified villages' (burgus), because in common speech people call the many dwelling-places established along the frontiers burgi. Hence also the nation of Burgundians got their name: formerly, when Germania was subdued, the Romans scattered them among their camps, and so they took their name from these places. }
\end{itemize}
\item [Livro: 13 - The cosmos and its parts (38.17\%)]\hfill
\begin{itemize}
\item[\textbf{isi\_en.13.XIII.9.Sall}]
\textit{9. There is a lake in the country of the Troglodytes; three times a day it becomes bitter, and then, just as often, sweet again. The Siloan spring at the foot of Mount Zion has no continuous flow of water, but bubbles forth at certain hours and days. In Judea a certain river used to go dry every Sabbath. }
\item[\textbf{isi\_en.13.XIX.4.Sall}]
\textit{4. People say that a lighted lamp floats on top, but when its light is extinguished, it sinks. This is also called the Salt Sea, or Lake Asphalti, that is, 'of bitumen,' and it is in Judea between Jericho and Zoara. In length it stretches 780 stades (i.e. about ninety miles) to Zoara in Arabia and its width is 150 stades, up to the neighborhood of Sodom. }
\item[\textbf{isi\_en.13.XV.1.Sall}]
\textit{1. Greek and Latin speakers so name the 'Ocean' (oceanus) because it goes around the globe (orbis) in the manner of a circle (circulus), [or from its speed, because it runs quickly (ocius)]. Again, because it gleams with a deep blue color like the sky: oceanus as if the word were mU?v?oç ("blue"). This is what encircles the edges of the land, advancing and receding with alternate tides, for when the winds blow over the deep, the Ocean either disgorges the seas or swallows them back. }
\end{itemize}
\item [Livro: 15 - Buildings and fields (28.57\%)]\hfill
\begin{itemize}
\item[\textbf{isi\_en.15.I.53.Sall}]
\textit{53. But Ascanius, after he had left his kingdom to his stepmother Lavinia, built Alba Longa. It was called Alba, 'White,' because of the color of a sow, and Longa because the town is elongated, in keeping with the great extent of the hill on which it is sited. From the name of this city the kings of the Albans took their names. }
\item[\textbf{isi\_en.15.II.31.Sall}]
\textit{31. The Capitolium of Rome is so called because it was the highest head (caput) of the Roman city and its religion. Others say that when Tarquinius Priscus was uncovering the foundations of the Capitolium in Rome, he found on the site of the foundation the head (caput) of a human marked with Etruscan writing, and hence he named it the Capitolium. }
\end{itemize}
\item [Livro: 11 - The human being and portents (3.14\%)]\hfill
\begin{itemize}
\item[\textbf{isi\_en.11.III.20.Sall}]
\textit{20. The Artabatitans of Ethiopia are said to walk on all fours, like cattle; none passes the age of forty. }
\end{itemize}
\end{description}
