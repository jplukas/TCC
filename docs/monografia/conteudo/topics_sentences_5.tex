\subsubsection{Tópico 5: \textit{law, laws, punishment, consuls, judge, justice, judicial, ius, charge, unjust}}
\label{tema:sentences_5}
\begin{description}
\footnotesize
\item [Livro: 5 - Laws and times (25.17\%)]\hfill
\begin{itemize}
\item[\textbf{isi\_en.5.I.3.S0}]
\textit{Numa Pompilius, who succeeded Romulus to the throne, first established laws for the Romans.}
\item[\textbf{isi\_en.5.II.1.S0}]
\textit{All laws are either divine or human.}
\item[\textbf{isi\_en.5.III.1.S0}]
\textit{Jurisprudence is a general term, and a law is an aspect of jurisprudence.}
\item[\textbf{isi\_en.5.IV.2.S1}]
\textit{Now this, or whatever is similar to it, is never unjust, but is held to be natural and fair.}
\item[\textbf{isi\_en.5.IX.1.S2}]
\textit{It concerns such things as legal inheritances, cretio (i.e. formal acceptance of an inheritance), guardianship, usucapio (i.e. acquisition of ownership by use): these laws are found among no other group of people, but are particular to the Romans and established for them alone.}
\end{itemize}
\item [Livro: 9 - Languages, nations, reigns, the military, citizens, family relationships (12.76\%)]\hfill
\begin{itemize}
\item[\textbf{isi\_en.9.I.7.S2}]
\textit{Then Mixed, which emerged in the Roman state after the wide expansion of the Empire, along with new customs and peoples, corrupted the integrity of speech with solecisms and barbarisms.}
\item[\textbf{isi\_en.9.III.6.S1}]
\textit{Because the Romans would not put up with the haughty domination of kings, they made a pair of consuls serve as the governing power year by year - for the arrogance of kings was not like the benevolence of a consul, but the haughtiness of a master.}
\item[\textbf{isi\_en.9.IV.24.S0}]
\textit{Curiales are the same as decurions, and they are called curiales because they 'have charge of ' (procurare) and carry out civic duties.}
\end{itemize}
\item [Livro: 2 - Rhetoric and dialectic (9.47\%)]\hfill
\begin{itemize}
\item[\textbf{isi\_en.2.IV.2.S0}]
\textit{. Judicial, in which a decision for punishment or reward is rendered according to the deed of that person.}
\item[\textbf{isi\_en.2.V.2.S2}]
\textit{Under purpose, judicial (iudicialis) and 'related to affairs' (negotialis).}
\item[\textbf{isi\_en.2.VIII.2.S0}]
\textit{The doubtful, in which either the judgment is doubtful, or a case is of partly decent and partly wicked matters, so that it arouses both benevolence and offense.}
\item[\textbf{isi\_en.2.X.6.S0}]
\textit{A law will be decent, just, enforceable, natural, in keeping with the custom of the country, appropriate to the place and time, needful, useful, and also clear - so that it does not hold anything that can deceive through obscurity - and for no private benefit, but for the common profit (communis utilitas) of the citizens.}
\item[\textbf{isi\_en.2.XII.3.S3}]
\textit{In words, when someone is said to have used words that are ugly and not appropriate to someone's authority, as if someone were to defame Cato the Censor himself as having incited young people to wickedness and lechery.}
\end{itemize}
\item [Livro: 18 - War and games (7.25\%)]\hfill
\begin{itemize}
\item[\textbf{isi\_en.18.XV.5.S0}]
\textit{A lawsuit consists either of argumentation or of evidence.}
\end{itemize}
\item [Livro: 10 - Vocabulary (6.40\%)]\hfill
\begin{itemize}
\item[\textbf{isi\_en.10.A.7.S1}]
\textit{Fair (aequus), meaning "naturally just," from 'equity' (aequitas), that is, after the idea of what is equal (aequus) - whence likewise 'equity' is so called after a certain equalness (aequalitate).}
\item[\textbf{isi\_en.10.C.61.S3}]
\textit{Confounded (confusus), so called from one's confession (confessio) of a wicked deed; hence also 'confounding' (confusio).}
\item[\textbf{isi\_en.10.D.80.S0}]
\textit{Condemned (damnatus) and condemnable (damnabilis): of these the former has already been sentenced, the latter can be sentenced.}
\item[\textbf{isi\_en.10.F.107.S0}]
\textit{Criminal (facinorosus), so called from the commission of a particular deed, for he does (facere) what harms (nocere) another.}
\item[\textbf{isi\_en.10.I.149.S1}]
\textit{Infitiator, "one who denies," because he does not confess (fateri) but strives against the truth witha lie.}
\end{itemize}
\end{description}
